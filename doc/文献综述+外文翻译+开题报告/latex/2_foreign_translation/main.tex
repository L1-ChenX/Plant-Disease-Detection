% !TeX program = xelatex
% !TEX root = main.tex
% !TeX encoding = UTF-8
\documentclass[UTF8,fontset=none,AutoFakeBold,AutoFakeSlant]{ctexbook}
\usepackage{titlesec}
\usepackage{geometry}
\usepackage{indentfirst}
\usepackage[hidelinks=false]{hyperref}
\usepackage{xcolor}
\usepackage{amsfonts}
\usepackage{url}
\usepackage{epsfig}
\usepackage{tikz}
\usepackage{amsmath}
\usepackage{subfigure}
\usepackage{multicol}
\usepackage{setspace}
\usepackage{caption}
\usepackage{float}
\usepackage{booktabs}
\usepackage{calc}
\usepackage{adjustbox}
\usepackage[ruled,vlined,linesnumbered]{algorithm2e} 
\usepackage{array}
\usepackage{threeparttable}
\usepackage{multirow}
\usepackage{bbm}
\usepackage{unicode-math}


% 字体设置
\setCJKfamilyfont{tnr_}{Times New Roman}
\setCJKfamilyfont{st_}{宋体}
\setCJKfamilyfont{fs_}{仿宋}
\setCJKmainfont{宋体}
\setmainfont{Times New Roman}
\setCJKfamilyfont{ht_}{黑体}
\setCJKfamilyfont{kt_}{楷体}

% 1.5倍行距
\linespread{1.5}
\geometry{a4paper,left=0.98in,right=0.98in,top=0.98in,bottom=0.98in}

\newcommand{\songti}{\CJKfamily{st_}}
\newcommand{\heiti}{\CJKfamily{ht_}}
\newcommand{\fangsong}{\CJKfamily{fs_}}
\newcommand{\kt}{\CJKfamily{kt_}}
\newcommand{\tnr}{\CJKfamily{tnr_}}

\newcommand{\zuozhe}{\fontsize{11pt}{16.5pt}\selectfont}
\newcommand{\xuexiao}{\fontsize{10pt}{15pt}\selectfont}

\pagestyle{plain}
\pagenumbering{arabic}

% 图片路径
\graphicspath{ {pictures} }

\captionsetup[table]{position=top}
\captionsetup[figure]{position=bottom}
\titlespacing{\section}{0pt}{1.5ex}{1.5ex}
\titlespacing{\subsection}{0pt}{1.2ex}{1.2ex}
\titlespacing{\subsubsection}{0pt}{1ex}{1ex}

\DeclareCaptionFormat{enformat}{#1\;#2#3}
\newcommand{\entitle}{
    \newpage
    \resetcounter
    \titleformat{\section}[block]{\centering\bfseries}{\thesection}{1em}{}
    \titleformat{\subsection}[block]{\fontsize{10.5pt}{10.5pt}\selectfont\bfseries}{\thesubsection}{1em}{}
    \titleformat{\subsubsection}[block]{\fontsize{10.5pt}{10.5pt}\selectfont}{\thesubsubsection)}{1em}{}
    \captionsetup[figure]{name=Fig.,labelsep=space,format=enformat}
    \captionsetup[table]{name=Table.,labelsep=space,format=enformat}
    \renewcommand{\algorithmcfname}{Algorithm}
}
\DeclareCaptionFormat{zhformat}{\songti\zihao{5}\textbf{\selectfont#1\;#2#3}}
\newcommand{\zhtitle}{
    \newpage
    \resetcounter
    \titleformat{\section}[block]{\bfseries}{\thesection.}{1em}{}
    \titleformat{\subsection}[block]{\zihao{-4}\bfseries}{\thesubsection}{1em}{}
    \titleformat{\subsubsection}[block]{\zihao{-4}}{\thesubsubsection)}{1em}{}
    \captionsetup[figure]{name=图,format=zhformat,labelsep=space}
    \captionsetup[table]{name=表,format=zhformat,labelsep=space}
    \renewcommand{\algorithmcfname}{算法}
}
\newcommand{\resetcounter}{
    \setcounter{page}{1}
    \setcounter{section}{0}
    \setcounter{subsection}{0}
    \setcounter{subsubsection}{0}
    \setcounter{figure}{0}
    \setcounter{table}{0}
    \setcounter{equation}{0}
    \setcounter{algocf}{0}
    \setcounter{footnote}{0}
}

% 在导言区或 chapter 前插入
\makeatletter
\renewcommand\thesection{\arabic{section}}       % 小节仅显示 1,2,3,...
\renewcommand\thesubsection{\thesection.\arabic{subsection}} % 二级标题 1.1, 1.2...
\makeatother


\begin{document}
\newlength\myheight
\newcommand\Mysavedprevdepth{}%
\newcommand\UnderlineCentered[3]{%
  \begin{adjustbox}{minipage=[t]{\dimexpr#1\relax},gstore totalheight=\myheight,margin=0pt}%
    \centering\leavevmode#3\par\xdef\Mysavedprevdepth{\the\prevdepth}%
  \end{adjustbox}%
  \hspace*{-\dimexpr#1\relax}%
  \begin{adjustbox}{minipage=[t][\myheight]{\dimexpr#1\relax},margin=0pt}%
    \vphantom{Eg}\lower\dimexpr#2\relax\hbox to\hsize{\leaders\hrule\hfill\kern0pt}\par
    \kern-\dimexpr#2\relax
    \xleaders\vbox to\baselineskip {\vfill\hbox{\lower\dimexpr#2\relax\hbox to\hsize{\leaders\hrule\hfill\kern0pt}}\kern-\dimexpr#2\relax}\vfill
    \kern\Mysavedprevdepth
  \end{adjustbox}%
}%
\newlength{\lwtm}
\setlength{\lwtm}{22mm}
\newlength{\remainingwidth}
\setlength{\remainingwidth}{35em}
% \setlength{\lwtm}{\widthof{原文1:}}
\newgeometry{left=0.58in,right=0.58in,top=0.98in,bottom=0.98in}

\thispagestyle{empty}
\vspace{42pt} 
\zihao{-2}
\begin{center}
    \textbf{\zihao{1}外\,文\,翻\,译}
    
\end{center}
\vspace{42pt}

\textbf{毕业设计题目:\UnderlineCentered{15.8em}{2mm}{基于深度学习的植物病害检测模型设计与移动应用系统开发实现}}

\vspace{64pt}

\begin{tabular}{p{\lwtm}p{\remainingwidth}}%
\textbf{原文1:} & \UnderlineCentered{\remainingwidth}{2mm}{\textbf{\textrm{Plant Disease Identification Using a Novel Convolutional Neural Network}}} \\
& \\[-14pt]
\textbf{译文1:} & \UnderlineCentered{\remainingwidth}{2mm}{\textbf{利用新型卷积神经网络的植物病害识别}} \\
\end{tabular}

\vspace{92pt}

\begin{tabular}{p{\lwtm}p{\remainingwidth}}%
\textbf{原文2:} & \UnderlineCentered{\remainingwidth}{2mm}{\textbf{\textrm{Recent advances in plant disease severity assessment using convolutional neural networks}}} \\
& \\[-14pt]
\textbf{译文2:} & \UnderlineCentered{\remainingwidth}{2mm}{\textbf{卷积神经网络在植物病害严重程度评估中的最新进展
}} \\
\end{tabular}

\restoregeometry

\entitle
\phantomsection
\addcontentsline{toc}{chapter}{原文1}
\noindent \zihao{4}\heiti 原文1\newline
\tnr

\begin{center}
    \zihao{4}\textbf{Spatial planning of urban communities via deep reinforcement learning}\\    
    \zuozhe Yu Zheng, Yuming Lin, Liang Zhao, Tinghai Wu, Depeng Jin,Yong Li\\
    \xuexiao Tsinghua University, Beijing, P. R. China.\\
    liyong07@tsinghua.edu.cn\\
\end{center}

\fontsize{10.5pt}{10.5pt}\selectfont

\noindent\textbf{Abstract: }Efective spatial planning of urban communities plays a critical role in the sustainable development of cities. Despite the convenience brought by geographic information systems and \
computer-aided design, determining the layout of land use and roads still heavily relies on human experts. Here we propose an artifcial intelligence urban-planning model to generate spatial plans for urban \
communities. To overcome the difculty of diverse and irregular urban geography, we construct a graph to describe the topology of cities in arbitrary forms and formulate urban planning as a sequential \
decision-making problem on the graph. To tackle the challenge of the vast solution space, we develop a reinforcement learning model based on graph neural networks. Experiments on both synthetic and real-world \
communities demonstrate that our computational model outperforms plans designed by human experts in objective metrics and that it can generate spatial plans responding to diferent circumstances and needs. We \
also propose a human–artifcial intelligence collaborative workfow of urban planning, in which human designers can substantially beneft from our model to be more productive, generating more efcient spatial plans \
with much less time. Our method demonstrates the great potential of computational urban planning and paves the way for more explorations in leveraging computational methodologies to solve challenging real-world \
problems in urban science.

\section{Methods}
\subsection{Problem formulation}
We formulate the problem of community spatial planning as a sequential Markov decision process (MDP), an interactive process between \
the planning agent and the environment, in which the agent observes \
the ‘state’ (the current conditions of the community), takes an ‘action’ \
(placement of an urban functionality) at each step and receives ‘reward’ \
(effect of the planned result) signaled by the environment, which \
undergoes a ‘transition’ (changes of the layout) according to the agent’s \
action. We utilize DRL to learn an effective policy that maps states to \
actions with a parameterized neural network. The neural network is \
optimized towards higher spatial efficiency through massive training \
under the MDP, with millions of training samples of the 4-tuple (state, \
action, reward and transition). As illustrated in Extended Data Fig. 1, \
our MDP is composed of two consecutive stages:

\begin{itemize}
    \item Land-use planning. Given the initial road conditions, the agent 
    places functionality blocks one at a time, either near existing 
    roads or near boundaries of previously placed land use. After 
    all the functionalities and open spaces are allocated, a reward 
    regarding the efciency of land use is returned to the agent, 
    which treats diferent land use as an integrated system. The fnal 
    land-use plan becomes the initial condition of road planning.
    \item Road planning. Boundaries of planned land use are viable locations for road construction. The agent builds roads iteratively, 
    turning one boundary into a road segment at a single step. 
    Stopped at a predefned termination step, a reward considering 
    the transportation efciency is returned to the agent.
\end{itemize}

The reward is only calculated at the last step of each stage to \
summarize the performance of land-use planning and road planning, \
respectively, and all intermediate steps receive a reward of 0. We define \
the reward for land use and road layout based on the 15-minute-city \
concept\
, which emphasizes the spatial efficiency to facilitate active \
transportation such as walking and cycling instead of automobiles. \
The two reward terms are calculated as follows:
\begin{equation}
    r_L = \alpha Service +Ecology,\label{land-use-planning}
\end{equation}
\begin{equation}
    r_R = Traffic,\label{road-planning}
\end{equation}
where Service measures the community life-circle index in the 15-minute city, Ecology measures the coverage of green space and parks, \
Traffic is a combination of road density and connectivity (Methods) \
and α serves as a hyper-parameter indicating the weight of service \
performance in the land-use reward. With the calculated reward values, \
we use proximal policy optimization to update the parameters of value \
and policy networks. We first train the agent for the land-use planning \
task until convergence and then train the agent to build roads with the \
optimal land-use plan obtained in the first stage. After two stages of \
training, the AI agent is able to design communities with an efficient \
spatial layout of both land use and roads.

\textbf{Graph model.}Different from previous Go and chip design tasks, \
urban planning is more challenging because of its much larger degrees \
of freedom in the problem form. Specifically, the conditions of previous tasks are regular, for example, placing stones on a 19 × 19 board or \
placing rectangular macros onto a grid chip canvas, which can be represented by pixels (raster). By contrast, the conditions for community \
spatial planning are diverse and irregular because the road corners and \
land blocks are usually not orthogonal. To accurately describe urban \
geographic elements, including land blocks (L), segments of roads and \
land-use boundaries (S) and junctions between roads and land-use \
boundaries (J), we use vector representations, which have been proven \
to have substantial advantages over raster representations in urban \
planning, and consist of the following three geometries:
\begin{itemize}
    \item  ‘Polygon’ that describes a vacant land to be planned (for \
    example, L1 in the right part of Extended Data Fig. 2a) or an \
    already planned land block (for example, L2 in the right part of \
    Extended Data Fig. 2a) with the coordinates of the land boundary;
    \item ‘LineString’ that represents a road segment (for example, S3 in 
    the right part of Extended Data Fig. 2a) or a boundary edge of a 
    land block (for example, S9 in the right part of Extended Data 
    Fig. 2a) with the coordinates of the start and end points; and
    \item ‘Point’ that stands for the junctions between roads and landblock boundaries (for example, J2 and J7 in the right part of \
    Extended Data Fig. 2a) with their coordinates.
\end{itemize}

We transform all geographic elements into the above three categories of geometries and then represent the whole community as a \
graph, in which nodes are the geometries and edges stand for the spatial \
contiguity relationship between these geometries, that is, two nodes \
are connected if the underlying two geometries touch each other. Each \
node stores its geographic information as the node features, including \
the type, coordinates, width, height, length and area of the geometry. \
In this way, spatial planning can be transformed as a problem of making \
choices on a dynamic graph (Extended Data Fig. 2), in which the graph \
evolves according to the agent’s actions.

In the land-use planning task, the agent selects one L–J edge that \
connects a vacant land and a junction, placing a given functionality at the \
location specified by the corresponding L and J (Extended Data Fig. 2a). \
In each step, the topology of the contiguity graph changes because \
the newly placed functionality generates new nodes and edges. New \
nodes include the new functionality itself, its boundaries, new junctions \
and split segments. New edges indicate the newly established spatial \
contiguity. Similarly, in the road planning task, the agent selects one \
S node that is currently a boundary and constructs it as a road segment \
(Extended Data Fig. 2b). Although the topology remains the same, \
the graph’s attributes alter because the selected node’s type changes \
from boundary to road. Through the problem reformulation with the \
graph model, we can now handle the irregular urban blocks and unify \
the two seemingly distinct stages of land use and road planning on \
one single graph.

\textbf{Action space design.}Another major challenge of urban planning is \
the huge action space, which is almost infinite in the original continuous space, and still too large in the reduced discrete graph space. The \
contiguity graph continues to grow as we place a functionality at each \
step, resulting in a large graph with thousands of nodes and edges. \
A typical spatial plan of a 2 km by 2 km community can take a total \
number of 100 planning steps in each stage, and the contiguity graph \
can have 4,000 edges and 1,000 nodes, which makes the action space \
$4,000^{100}$ and $1,000^{100}$ for the two stages, respectively. In addition, valid \
actions are extremely sparse in the space, and a substantial portion \
of actions is of low quality and will lead to unreasonable results, such \
as placing a facility in the center of a vacant land without connecting \
roads. Therefore, it is crucial to reduce the action space and avoid \
unreasonable actions.

To address this challenge, we propose a general DRL framework in \
which an intelligent agent perceives and makes decisions in a reduced \
graph space, and the environment handles urban elements in the original geographic space and generates graph states according to the geographic spatial layout. Meanwhile, we decompose the entire action \
space into a Cartesian product of three sub-spaces, including what \
to plan, where to plan and how to plan, and let the DRL agent focus on \
the core issue of where to plan. The first sub-space of what to plan can \
be eliminated by fixing the planning order of different land-use types \
through domain knowledge, allowing land-use types that are more \
dependent on the initial road network to be planned earlier (Methods). \
To avoid apparently improper actions in where to plan, we impose \
planning constraints on the agent’s actions, with an action mask that \
blocks out unreasonable options, that is, only L–J edges and S nodes are \
candidates for the two planning stages, respectively. After selecting one \
L–J edge for a given land-use type, the functionality is placed in the corresponding land block (L node) at the location of the corresponding junction ( J node), whose shape and size are determined by predefined rules \
that maximize the reuse of existing roads and boundaries (Methods); \
thus, the last sub-space of how to plan is effectively eliminated. Through \
these designs, we narrow the action space to a solvable scale and filter \
out most unreasonable actions, enabling efficient optimization for \
DRL algorithms. In summary, the original problem of spatial planning \
is successfully transformed into a standard sequential decision-making \
process on a graph with moderate action space.

\textbf{Framework.}After the above problem reformulation and action space \
design, we propose a DRL framework in which an AI agent learns to \
lay out land use and roads by interacting with the spatial planning \
environment, as illustrated in Extended Data Fig. 3. The sequential \
MDP (Extended Data Fig. 3e,f) contains the following key components:
\begin{itemize}
    \item States summarize the current spatial plan with the previously 
    introduced contiguity graph containing rich node features, and 
    other information, such as statistics of diferent land use types.
    \item Actions indicate the locations to place the current land use or 
    construct a new road segment, which are transformed from the 
    selected edges or nodes in the contiguity graph.
    \item Rewards are 0 for all intermediate steps, except for the last step 
    in each stage, in which it evaluates the spatial efciency of land 
    use and roads.
    \item Transitions describe the changes of the layout given the \
    selected location, and the transitions occur in both the original \
    geographic space (new land use and road on the map) and the \
    transformed graph space (new topology and attributes of the \
    graph).
\end{itemize}

At each step, the agent represents the state by encoding the graph \
with a GNN. Via multiple message passing and non-linear activation \
layers, the GNN state encoder generates effective representations of \
edges, nodes and the whole graph (Extended Data Fig. 3a), which will be \
leveraged by the value and policy networks (Extended Data Fig. 3b–d). \
Specifically, because choosing locations for land use is equivalent to \
selecting edges on the graph, the land-use policy network takes the \
edge embeddings and scores each edge with an edge-ranking MLP, \
as shown in Extended Data Fig. 3b. The obtained score for each edge \
indicates the sampling probability of the corresponding edge, which \
is returned to the environment and becomes the probability of placing \
the land use at the location specified by that edge. Similarly, in road \
planning, the road policy network takes node embeddings and scores \
each node with a node-ranking MLP (Extended Data Fig. 3d), outputting \
the probability of choosing one land block boundary and building it as \
a road segment. Finally, the value network takes in the graph embedding that summarizes the whole community and predicts the planning \
rewards with a fully connected layer (Extended Data Fig. 3c). To master \
the skills of spatial planning, millions of spatial plans are accomplished \
by the proposed model to search the large solution space during the \
training process, which is utilized as real-time training data to update \
the parameters of the neural network.\\
\subsection{Detailed methodology}

\textbf{Framework.}As introduced in the paper, we use vector geometries \
including Polygon, LineString and Point to describe urban geographic \
elements. Specifically, there are ten types of land blocks that are represented as Polygon, including the initial vacant land to be planned, and \
nine different functionality types, which are residential (RZ), school \
(SC), hospital (HO), clinic (CL), business (BU), office (OF), recreation \
(RE), park (PA) and open space (OP). In addition, there are two types \
of segments (roads and land-use boundaries) that are represented by \
LineString and one type of junction (intersections between roads and \
land-use boundaries) that is represented by Point. Therefore, a community is faithfully represented by a table of geometries, in which each row \
is a geographic element with three columns of ID, type and geometry. \
Initial conditions of a community consist of all the original land blocks, \
roads and intersections, whose accurate coordinates are recorded \
by their corresponding geometries in the table of geometries. In the \
synthetic grid community, we experiment on a basic community with \
a size of 2.4 km by 2.4 km, containing 16 rectangular vacant lands, 40 \
horizontal or vertical initial road segments and 25 road intersections, \
as shown in the first step of Extended Data Fig. 1. In the real-world community, we replicate the road network of HLG and DHM communities in \
Beijing from OpenStreetMap using OSMnx30 and geopandas, reserve \
residential blocks and leave other areas as vacant land to be renovated. \
Finally, we obtain two communities of around 4 km\textsuperscript{2} as shown in Fig. 1a\
and Supplementary Fig. 11a.

\textbf{Planning needs and requirements.}Before carrying out the actual \
spatial planning, we need to determine the planning needs and requirements, which serve as the configuration of the planning environment. \
The planning need describes the amount that each land-use type has\
to achieve, either in area or in number, for example, residential blocks \
of 50\% community area and three hospitals. Meanwhile, we also have \
requirements on the minimum area (in square meters) of each planned \
block; for example, the area of one school is at least 10,000 m\textsuperscript{2}\
. Supplementary Table 3 shows an example of the planning needs and requirements for a community, in which 15\% of the community area needs to \
be planned as parks; thus, it serves as a green community. Only spatial \
plans that satisfy all the needs and requirements are considered as \
successful episodes and reserved as training samples, and those failed \
episodes are discarded. In our framework, the planning needs and \
requirements are configurations for the environment, making our \
model highly flexible in generating spatial plans. Specifically, once \
we obtain a well-trained model under one configuration, we can simply change the configuration and directly perform model inference \
without re-training to generate plans for different planning needs \
and requirements, such as the community plans of different service \
supplies in Fig. 1c,d.

\textbf{Planning order of land-use types.}As introduced in the paper, in order \
to reduce the huge action space, we fix the planning order of different \
land-use types based on domain knowledge and make the agent focus \
on the core task of selecting locations. Because feasible locations \
are next to existing junctions, land use that is planned earlier will be \
closer to initial roads with more convenient traffic. Therefore, we first \
plan those facilities that depend more on roads, including hospitals \
(clinics), schools and recreation. Meanwhile, at later steps of land-use \
planning, the shape of feasible vacant lands tends to be more irregular \
and fragmented, which is not suitable for residential blocks that usually \
occupy a whole plot of land; thus, we distribute residential blocks after \
planning the above road-dependent facilities. Finally, we arrange those \
land-use types that are not much demanding in land shapes. After all the \
planning needs are satisfied, the remaining vacant lands are assigned \
as open spaces. In summary, the planning order in our framework is \
fixed as follows: hospital, school, clinic, recreation, residential, park, \
office, business and open space. Letting the agent determine the order \
of land-use types may be an alternative approach. However, it will make \
the problem much more complicated, as the action space is increased \
drastically. In practice, our fixed order generates sound spatial plans.

\textbf{Land cutting.}In land-use planning, the environment receives the \
action from the agent, which is the selected L–J edge, and cuts a new \
land from the corresponding land block (L node) at the location of the \
corresponding junction ( J node). We develop a rule-based system with \
expert knowledge incorporated to determine the shape and size of the \
new land. The rule-based system is roughly composed of three steps: \
(1) Determine the relationship between J and L, such as in the middle of \
a road or at the corner. (2) Determine the reference line along existing \
boundaries from junction J, which can be I-shape, L-shape and U-shape. \
(3) Determine the length of inward extension from the reference line \
into the block L, forming the final sliced new land. The three steps \
are conducted according to expert knowledge, in order to meet the \
planning requirements and fit the current plan as closely as possible.

\textbf{State.}Our state contains three parts: (1) urban contiguity graph, \
(2) current object to be placed and (3) community statistics. We construct \
a graph to represent the current community information as illustrated \
in Extended Data Fig. 2, in which nodes are urban geographic elements \
and edges indicate the spatial contiguity relationship. We compute \
rich geographic attributes as node features, including the type, coordinates, area, length, width and height of the underlying urban element. \
The edges are represented by a sparse adjacency matrix. As for the \
current object to be placed, its type is determined by the environment \
according to planning needs and planning order, that is, the environment will traverse the planning order and transit to the next type if the \
planning needs for the previous type have been satisfied. We treat the \
current object as a virtual isolated node, with its type feature provided \
by the environment and other node features left as default values. \
Lastly, community statistics include the area and count of different \
land-use types in the current plan, as well as the planning needs, which \
summarize the current conditions and the progress of spatial planning.

\textbf{Action.}As illustrated in Extended Data Fig. 2a, land-use planning is \
reformulated as a sequential MDP in which the agent selects an edge in \
a dynamic graph. Therefore, the action space for land-use planning is \
the probability distribution of choosing from N edges, and we sample \
from this distribution to obtain the action. Similarly, road planning is a \
sequential MDP of choosing nodes as shown in Extended Data Fig. 2b; \
thus, the action space for road planning is the probability distribution \
over M nodes, which is sampled to generate the node selection action. \
In addition, as introduced previously, we impose constraints on the \
action space; for example, the agent can only select L–J edges (between \
vacant lands and junctions) and S nodes (land-use boundaries) to \
avoid unreasonable spatial plans. Thus, we calculate a mask in each \
step that indicates feasible options, and the probability distribution \
will be multiplied by the mask, allowing only feasible edges or nodes \
to be sampled as actions.

\textbf{Policy and value networks.}As shown in Extended Data Fig. 3b–d, \
we develop separate policy networks to take actions in the policy \
and value stages, respectively, as well as a value network to predict \
the performance of spatial plans. The three networks share the same \
state encoder to obtain state representations, taking full advantage \
of the GNN. Policy networks generate the probability distribution by \
scoring the edges and nodes in the graph, and then sample from this \
distribution to take actions. Meanwhile, the value network evaluates \
the whole graph to predict spatial efficiency, providing feedback for \
the community plan. In this section, we introduce the detailed design \
of the three networks.

\textbf{Land-use policy network.}In land-use planning, the agent places the \
current object at the location specified by the selected edge. The effect \
of edge selection is related to both the edge and the current object; for \
example, placing a hospital next to an already planned hospital may \
lead to low service efficiency. Therefore, the land-use policy network \
considers both the edge and the current object as input. As shown \
in Extended Data Fig. 3b, a feed-forward network, which is an edgeranking MLP, is developed to score each edge:
\begin{equation}
    s(e_{ij})=FF_{land}(e_{ij}^L||v_c||e_{ij}^L-v_c||e_{ij}^L·v_c),\label{each-edge-score}
\end{equation}
here the difference and the inner product of $e^L_{ij}$ and $v_c$ are also concatenated to emphasize the relationship between the current object to 
be planned and those already planned land use. The scores are converted to a probability distribution over all edges using softmax:
\begin{equation}
    Prob(e_{ij})=\frac{e^{S(e_{ij})}}{\sum_{s,t\in E}e^{s(e_{st})}},\label{scores-converted-to-probability-distribution}
\end{equation}
which is sampled to select an edge.

\textbf{Road policy network.}In road planning, the agent selects one boundary \
node and plans a road at its location. Different from that of land-use \
planning, the topology of the graph is stable with no new nodes to be \
added. Thus, there is no need to include the current object. Meanwhile,\
the road policy network takes node embeddings as input, which already \
contain neighbor geographic information through message passing of \
GNN. As in Extended Data Fig. 3d, another node-ranking feed-forward \
MLP is adopted to score each node:
\begin{equation}
    s(v_i) = FF_road(v_i).\label{score-based-in-node-ranking-feed-forward-MLP}
\end{equation}
The score is also transformed to probability with a softmax operator:
\begin{equation}
    Prob(v_{ij})=\frac{e^{S(v_{ij})}}{\sum_{j\in N}e^{s(v_{j})}},\label{score-transformed-to-probability}
\end{equation}
and we sample from this probability distribution to select one node.

\textbf{Value network.}As shown in Extended Data Fig. 3c, we develop a value \
network to judge the current planning situations and predict the planning \
performance. Because it is an overall evaluation of the entire community, \
we take the graph-level representation as the input of the value \
network. Meanwhile, we also include community statistics. Specifically, \
we concatenate graph representations and the statistics embedding, \
and adopt a fully connected layer to predict the performance:
\begin{equation}
    \hat{v} = fa(g^L||h_s),\label{predict-performance}
\end{equation}
where $\hat{v}$ is the estimated value of the current plan.

\textbf{Reward.}We train the policy networks to optimize the efficiency of \
spatial layout, with respect to service, ecology and traffic. As in \
equations (1) and (2) of this paper, we define reward functions that give a \
comprehensive evaluation of the above metrics. Meanwhile, the reward \
values can be quickly computed within tens of milliseconds given a \
spatial plan, making it possible to collect large-scale samples for \
training DRL models. In this section, we introduce how the three metrics are \
calculated. It is worth noting that our framework is flexible and can be \
extended to include more metrics in the reward.

\textbf{Service.}We adopt the concept of 15-minute life circle, which requires \
that basic services of the community be reachable for residents within \
15 min by walking or cycling. Specifically, as demonstrated in Fig. 1b,c, \
we consider five different basic services, each of which is related to \
one or two facilities, that is, education (school), medical care \
(hospital, clinic), working (office), shopping (business) and entertainment \
(recreation). Therefore, the 15-minute life circle means that the distances \
between facilities and residential zones need to be less than the \
walking distance of 15 min, which is set as 500 m in our experiments. \
We define the service metric as the proportion of accessible services \
within 500 m, and the metric is averaged for all residential zones. \
Formally, given a community spatial plan p, the service metric is \
calculated as follows:
\begin{equation}
    d(i,j) = min\{EucDis(RZ_i,FA_1^j),\dots,EucDis(RZ_i,FA_{n_j}^j)\},\label{min-distance-for-the-ith-residential-zone}
\end{equation}
\begin{equation}
    Service_i = \frac{1}{5}\sum_{j=1}^{5}\mathbbm{1}[d(i,j)<500],\label{life-circle-metric-for-the-ith-residential-zone}
\end{equation}
\begin{equation}
    Service = \frac{1}{n_{RZ}}\sum_{i=1}^{n_{RZ}}Service_i,\label{life-circle-metric-for-all-residential-zones}
\end{equation}
where $EucDis$ is the Euclidean distance, $d(i,j)$ is the minimum distance \
for the $i$th residential zone $RZ_i$ to access the $j$th service that is provided \
by facility $FA^j$ and $n_j$ is the total number of facilities $FA^j$. $Service_i$ is the \
15-min life-circle metric for the ith residential zone, and we average over \
all $n_{RZ}$ residential zones to obtain the final service metric for the whole \
community. This service metric guides the agent to arrange facilities\
in a more decentralized way and close to residential zones, which is \
critical for increasing the ability of community services.

\textbf{Ecology.}The ecology of a community is important to the physical and \
mental health of residents; thus, we include an ecology metric that 
measures the layout efficiency of parks and open spaces. In general, 
parks and open spaces serve the residents who live in the neighborhood, \
and we hope they can serve as many residential areas as possible. \
Formally, we define the ecological serving range as the region within \
300 m from a park or an open space, and the ecology metric measures \
the proportion of residential areas that are covered by the ecological \
serving range. The metric is calculated as follows:
\begin{equation}
    \begin{aligned}
        ESR = & Union\{Buffer(PA_1,300),\dots,Buffer(PA_{n_{PA}},300),\\
        & Buffer(OS_1,300),\dots,Buffer(OS_{n_{OS}},300)\},\label{ecological-serving-range}
    \end{aligned}
\end{equation}
\begin{equation}
    A_{RZ} = \sum_{i=1}^{n_{RZ}}Area(RZ_i),\label{all-residential-areas}
\end{equation}
\begin{equation}
    A_{RZ}^e = \sum_{i=1}^{n_{RZ}}Area(Intersection(RZ_i,ESR)),\label{intersection-between-all-parks-and-open-spaces}
\end{equation}
\begin{equation}
    Ecology = \frac{A_{RZ}^e}{A_{RZ}},\label{ecology}
\end{equation}
where $Buffer(PA_i,300)$ and $Buffer(OS_i,300)$ represent the regions that \
extend the park and open space 300 m outward, which is their serving \
range, and $ESR$ is the ecological serving range that combines the serving \
range of all parks and open spaces. The ecology metric encourages the agent \
to maximize $A^e_{RZ}$; thus, the greenness of the community plan is promoted.

\textbf{Traffic.}For the second stage of road planning, we evaluate the traffic 
efficiency from three perspectives, including density, connectivity
and spacing. Road density is the ratio of the total length of roads to 
the land area. Connectivity is a network characteristic that reflects the 
strength of how different parts of a network are linked with each other, 
and we choose the number of connected components and the number 
of dead-end roads. To achieve appropriate road spacing, we also include 
two terms to penalize too large (>600 m) and too small (<100 m) spacing. 
Formally, the traffic metric is calculated as follows given the road 
plan $p_R$ and the converted graph $g_R$ from the planned road network:
\begin{equation}
    T_{density}=\frac{Length(p_{R})}{A_c},\label{density}
\end{equation}
\begin{equation}
    T_{connectivity}=\frac{1}{NCC(g_R)}+\frac{1}{1+\sum_{v \in g_R}\mathbbm 1[Degree(v)=1]'},\label{connectivity}
\end{equation}
\begin{equation}
    T_{spacing}=\frac{1}{1+\sum_{r \in p_R}\mathbbm 1[Length(v)>600]}+\frac{1}{1+\sum_{r \in p_R}\mathbbm 1[Length(v)<100]'},\label{spacing}
\end{equation}
\begin{equation}
    Traffic=\frac{1}{3}*(T_{density}+T_{connectivity}+T_{spacing}),\label{traffic}
\end{equation}

where Length calculates the length of a road segment, $A_c$ is the area \
of the community, $NCC$ calculates the number of connected components \
in a network and Degree calculates the degree of a node in the graph. \
Combining the three perspectives, the traffic metric encourages the \
agent to plan denser roads and, at the same time, guarantees \
connectivity and appropriate spacing, without creating dead-end roads \
or planning too-long or too-short road segments.

\textbf{Model training.}We train our model for hundreds of iterations \
to learn the skills of spatial planning. In each iteration, we collect \
training samples of a few thousand episodes and update the parameters of \
our model using proximal policy optimization. Specifically, the loss \
function is a combination of policy loss, policy entropy and value loss. \
Policy loss is a surrogate clipped objective to improve the policy with \
safe exploration, which is calculated as follows:
\begin{equation}
    r_t(\theta) = \frac{\pi_\theta(a_t|s_t)}{\pi_{\theta_{old}}(a_t|s_t)},\label{ratio-of-new-old-policy}
\end{equation}
\begin{equation}
    L_{policy} = min(r_t(\theta)\hat{A_t},clip(r_t(\theta),1-\epsilon,1+\epsilon)\hat{A_t}),\label{policy-loss}
\end{equation}
\begin{equation}
    \hat{A_t} = Q(s_t,a_t)-V(s_t),\label{advantage-function}
\end{equation}
where $\theta$ is the parameters of our model, $r_t(\theta)$ is the ratio of the \
probability of the new policy to the old policy, $\hat{A_t}$ is the advantage \
function and clip restricts the update to be not too large. The entropy loss controls \
the balance between exploitation and exploration, which is calculated as follows:
\begin{equation}
    L_{entropy} = Entropy[Prob(a_1),\dots,Prob(a_{n_a})],\label{entropy-loss}
\end{equation}
where $n_a$ is the total number of actions that equals to $M$(edges) or $N$
(nodes) in different planning stages, and Prob is obtained by policy 
networks according to equations (12) and (14). We use mean squared 
error loss to supervise the value prediction:
\begin{equation}
    L_{value} = MSE(\hat{v_t},R_t),\label{value-loss}
\end{equation}
where $R_t$ is the return value from ground-truth and $\hat{v_t}$ is estimated \
by value network according to equation (15). The final loss function is a \
weighted sum of the above three terms:
\begin{equation}
    L = L_{policy}+\beta L_{entropy}+\gamma L_{value},\label{sum-Loss}
\end{equation}
where $\beta$ and $\gamma$ are hyper-parameters in our model.

\textbf{Model inference.} After we obtain a well-trained model, we perform \
model inference to generate community plans. We use the policy \
networks and compute the probability distribution over different actions \
according to equations (12) and (14), that is, the probability of selecting \
different edges and nodes. Then the most likely action is chosen to place \
land use or road at the location specified by the action:
\begin{equation}
    a = \text{argmax}\{Prob(a1),\dots,Prob(a_{n_{a}})\},\label{action}
\end{equation}
where $n_{a}$ is $M$ or $N$ for land use and road planning, respectively. It is\ 
worth noting that we can directly perform model inference under a different \
setup without re-training, and the results are illustrated in Fig. 1c,d.

\textbf{Integration with manually designed planning concepts.} The DRL \
framework is not designed for replacing human designers but serves as an \
intelligent assistant to improve the productivity of human designers. \
Specifically, AI models are good at optimizing spatial efficiency in large \
solution spaces, whereas human designers are good at conceptual prototyping. \
Therefore, we design a new workflow, in which human and AI collaboratively \
accomplish urban-planning tasks and leverage their respective expertise. \
As shown in Supplementary Fig. 7a, we propose a workflow with four key \
steps of conceptualization,planning, adjustment and evaluation, where AI \
takes responsibility for the planning step. In the workflow, human designers \
can leave the heavy and specific planning work to the AI, and they only need \
to provide relatively abstract conceptual planning and make adjustments to \
the spatial plans generated by the AI. We represent planning concepts \
as two major types, center and axis, and each concept is related to one \
or several land-use functions. For example, in the left part of Supplementary \
Fig. 7b, the RE center in the HLG community represents the \
concept that encourages recreation zones near the specified location. \
Similarly, in the right part of Supplementary Fig. 7b, the BU\&OF axis in \
the DHM community represents the concept that expects a business \
and office core along the specified band region. We feed the initial \
conditions of the community and the planning concepts to the model, \
and then train our DRL model to realize the planning concepts while at \
the same time optimizing spatial efficiency.
As the concept of center and axis is essentially the spatial relationship \
between specific land-use functions and predefined locations, it \
can be easily integrated into our framework. Specifically, we utilize \
customized reward functions to implement planning concepts, that \
is, we add a reward to reflect the extent of consistency with the \
planning concept. For the center concept, we calculate the fraction of \
concept-related land-use functions in the region near the specified \
center location as follows:
\begin{equation}
    r_c = \frac{1}{n_c}\sum_{j=1}^{n_c}\mathbbm{1}[T_j\in T_c],\label{concept-related-land-use-functions}
\end{equation}
where $L_a$ is the length of the axis, $L^p_a$ is the distance of projected points \
of concept-related land-use functions on the axis, $n_a$ is the number of \
land use blocks within 100m from the predefined axis and $T_a$ is the \
land-use function related to the concept. This reward encourages the \
DRL agent to place concept-related land-use functions as evenly as \
possible in the band area around the axis. The concept reward is \
combined with efficiency reward, including service and ecology, by \
weighted sum. Through jointly optimizing efficiency and concept \
rewards, the DRL agent learns to improve spatial efficiency on the basis \
of realizing predefined planning concepts.

\zhtitle
\phantomsection
\addcontentsline{toc}{chapter}{译文1}
\noindent \zihao{4}\heiti 译文1\newline
\renewcommand{\fig}[3]{
    \begin{figure}[!htb]
        \centering
        \includegraphics[width=#2\textwidth]{#1.png}
        \caption{#3}
        \label{#1-zh}
    \end{figure}
}

\zihao{-4}\songti
\noindent 作者:Parag J. Siddique, Kevin R. Gue, John S. Usher\\
\noindent 出处:Transportation Research Part C: Emerging Technologies, Volume 127, June 2021, 103112
\vspace{2ex}

\begin{center}
    \zihao{4}\textbf{基于谜题的停车}
\end{center}


\noindent\textbf{摘要:} 在普通停车场中,大部分空间并非用于停车,而是用于车辆在停车位之间行驶的车道。车道不能被阻塞,一个简单的原因是:被阻塞的车可能需要在挡住它的车之前离开。自动停车和自动驾驶车辆的智能通信能力为克服这一限制提供了机会,因此可以实现更高的汽车存储容量。我们展示了如何通过集中控制器将干扰的车辆移开,从而最大化停车场中的汽车数量。我们提供了单入口小型停车场的最优结果,并为更大停车场提供了启发式方法。停车容量的提升可达80\%。

\section{基于谜题的停车场}
自主乘用车辆的前景为改善城市交通提供了许多有趣的机会。在停车方面,远程控制汽车的潜力允许汽车停放得更近,因为不需要打开车门让乘客下车。停车密度的提升潜力估计为20\%。另一个想法是,自动驾驶车辆不需要停在繁忙的城市区域。相反,可以在廉价的郊区创建远程停车区域。另一个被广泛接受的观点是,自动驾驶车辆将消除私人车辆所有权。自动出租车将带来共享出行,从而大幅减少停车需求。

在这一背景下,博世与梅赛德斯和福特合作,设计了一座为自动驾驶车辆设计的自动泊车场。到达停车场时,乘客在一个送车点离开车辆。停车场配备了智能传感系统,引导车辆找到停车位。每当乘客需要车时,可以通过智能手机应用程序检索车辆,车辆将到达取车点。值得注意的是,不需要人工控制汽车,也无需建设额外的市政基础设施。

在这样的自动化停车场中,通过消除行车道,可以实现更高的停车密度。在普通停车场中,行车道不能被封锁的一个简单原因是:被封锁的车可能需要在挡住它的车之前离开。然而,自动泊车场为克服这个问题,实现更高停车密度提供了机会。因此,我们已经拥有实现更高密度所需的技术,一切需要高效的设施设计和规划。

\fig{f1}{0.8}{其他作者提出的高密度停车设计。 (a) Ferreira等人 (2014) 提出了一种通过在平行车道上相互阻挡的设计来停放汽车。他们在设施的入口和出口都分配了缓冲区域,以便车辆可以进行必要的移动进行重新定位。 (b) Timpner等人在被阻挡的车道之间放置了垂直过道,以减少干扰车辆的数量。阻挡的车辆重新定位到过道上,为即将离开的车辆清出道路。 (c) Banzhaf等人通过添加一个车辆阻挡的过道来修改现有的停车场。 (d) 在Nourinejad等人的设计中,过道的宽度随着车道的深度而变化,以帮助容纳重新定位的车辆。}

一些作者设想了一些停车方案,以提高停放汽车的密度,或等效地提高停车场的容量。Ferreira等人首次提出了自主车辆的高密度停车方案。在他们的设计中,汽车通过在平行车道上相互阻挡来停放,类似于材料处理系统中使用的深车道存储系统(图\ref{f1-zh}a)。他们假设有一个自主停车控制器,负责控制停车场内的车辆移动。这一组研究人员已将这项工作扩展到考虑各种操作参数,如每辆车的停车空间、操纵次数、行驶距离和检索时间。他们的设计将停车容量提高了50\%。

为了减少干扰车辆的数量,Timpner等人修改了Ferreira等人的设计,减小了车道深度,并允许车辆从车道的任一端离开。作者估计密度提高了33\%。Banzhaf等人建议通过在现有过道内允许垂直的阻挡车道来修改现有布局。他们报告密度增加了25\%。Nourinejad等人描述了一个类似的系统,其中过道的宽度随着车道的深度变化,以帮助容纳重新定位的车辆。作者制定了一个混合整数非线性规划来最小化预期的车辆重新定位次数。然而,所有这些研究人员都计划针对更适合郊区地区的大型停车场。

从文献中我们了解到,自主车辆不会在停车场闲置。在放下乘客后,它将用于运送另一名乘客。然而,由于供需之间存在差距,当没有行程时会有一些空闲时间,因此车辆必须停在某个地方。如果这些车辆去郊区停车,它们将无法响应客户的呼叫。因此,自主车辆不会选择去城外停车,而是会在城市中心周围巡游以打发时间,从而导致城市拥堵。因此,我们无法完全消除城市停车。相反,我们可以寻找使停车更容易访问和更便宜的方法。为此,我们可以为城市区域提供小型而密集的停车场。这些停车场可以与零售业中的微型履行中心相类比。微型履行中心是靠近客户的小型仓库,专为按需电子商务履行而设计。每当自主车辆处于空闲状态时,它可以停在这些小型停车场。由于这些停车场将位于城市区域,因此每当车辆接收到行程请求时,它可以迅速服务请求。作为增值服务:在这些停车场中还可以引入车辆充电、清洁或类似的服务。

在本文中,我们研究了自主车辆停车场的高密度布局设计。我们的工作在两个重要方面与现有文献有所不同:

\begin{enumerate}
    \item 我们为小型停车场开发最优设计,这是人们可能在城市环境中找到的。
    \item 所有现有的研究都是基于一个布局,然后研究在该布局下的最优参数和性能。我们提出并回答了更基本的问题,即在不假设任何特定布局或结构的情况下,停车场可以停放多少辆车。正如我们将展示的,这种放宽导致了具有最大密度的设计,而不采用传统的“排和过道”停车结构。
\end{enumerate}

\fig{f2}{0.5}{Rush Hour难题。游戏的目标是通过移动其他车辆让红车直接移动到对面的出口。}

\fig{f3}{0.9}{(a) 一个$4 \times 4$停车场的示例,其中有两辆车,一辆水平停放在单元格(43, 44),另一辆垂直停放在单元格(23, 34),而单元格(11, 21)是输入/输出点。(b) 停车场内的三个有效移动。每个移动根据所行单元格的数量分配一个权重。}

我们处理的问题类似于Rush Hour难题(图\ref{f2-zh}),这在理论计算机科学领域有相当多的文献。Flake和Baum表明,具有$n \times n$网格的Rush Hour的广义版本是PSPACE-complete。Tromp和Cilibrasi以及Hearn和Demaine表明,仅包含汽车(没有$3 \times 1$的卡车)的Rush Hour也是PSPACE-complete。我们的问题在很多方面都与Rush Hour不同:游戏只允许车辆前后移动,而真实汽车可以转弯;而Rush Hour的目标是移走一个特定的车辆,而真实的停车场必须允许任何请求的车辆到达或离开。在物料处理文献中,也已经讨论了没有定义过道的高密度存储,其中使用可以沿着四个基本方向移动的输送模块存储和检索单元负载或容器。然而,在我们的停车场问题中,汽车独自移动,不需要任何输送模块的帮助,而且汽车可以朝不同的方向行驶。

\section{建模基于谜题的停车场}

我们将停车场视为一个矩形网格。汽车可以水平或垂直停放。我们假设汽车在网格中占据两个单元格,这意味着汽车的大小为$1 \times 2$。停车场的左下角有一个入口/出口点。我们称这个点为输入-输出(I/O)点。在图\ref{f3-zh}a中,两辆车停在一个$4 \times 4$停车场中,一辆停在单元格(43, 44),另一辆停在单元格(24, 34)。单元格(11, 21)表示I/O点。

\fig{f4}{1}{生成$4 \times 4$停车场状态空间图的过程。 (a) 单车图。大红色节点是根节点—这个图中的第一个节点。绿色节点是开放节点。开放节点总是允许新车进入停车场。根节点和一个开放节点都已注释。 (b) $4 \times 4$网格中的2车图。红色节点是初始节点,包括I/O单元格。一个初始节点已注释。 (c) 连接单车图和2车图。单车图中的开放节点连接到2车图中的初始节点,用蓝色突出显示。这些连接器称为桥接边缘。先前注释的节点的位置也显示出来。}

车辆在停车场内可以进行三种类型的移动:直行、直角和平行,这些操作可以前进或后退完成(图\ref{f3-zh}b)。假设车辆花费一个单位的时间在一个单元格上行驶,可以为每个移动分配一个权重以量化行驶时间。例如,车辆进行直行时移动一个单元格,因此将此移动分配一个权重为一。类似地,车辆进行直角和平行移动时移动四个单元格,因此将这些移动分配一个权重为四。我们假设停车场中有一个中央控制器代理,可以与车辆通信以执行所有移动。

读者可能会注意到这里我们并没有指定单元格的尺寸。单元格的大小是根据车辆的尺寸和运动学来决定的。相反,我们希望将单元格的设计留给专业人士。

\subsection{停车场能容纳多少辆车?}

我们使用状态空间图$G = (V, E)$对停车场中的车辆进行建模,其中$V$是表示停车场配置的顶点集合,$E$是表示配置之间可行转换的边缘集合。在本文中,顶点也可以用术语“节点”来表示。配置被定义为停车场的尺寸,停放的车辆数量以及它们的位置。术语“状态”也被用作配置的替代术语。我们想知道:停车场能停放多少辆车?为了开始解决这个问题,我们解决一个小例子,即$4 \times 4$停车场。

在一个$4 \times 4$停车场中,有十六个单元格,每辆车占用两个单元格。因此,我们最多可以放置八辆车。我们首先为$4 \times 4$网格生成状态空间图。在不考虑可操纵性的情况下,一个单独的车辆可以放置在24个不同的位置,每个位置都代表图中的一个潜在节点。然后,我们对所有潜在节点执行两两搜索,确定它们是否可以通过可行的操纵连接。我们将结果称为单车图(图\ref{f4-zh}a)。接下来,我们以每种可能的方式放置两辆车,以创建2车图。然后我们消除了具有重叠车辆的配置。例如,配置[(11, 21), (21, 22)]是不可行的,因为单元格21被两辆车使用,这在物理上是不可能的。我们还消除了占据整行或整列的配置,因为它们是不可行的(除了[(11, 21), (31, 41)])。最后,我们使用它们各自的移动连接节点,构建了完整的2车图(图\ref{f4-zh}c)。同样,我们可以继续构建多达八辆车的图。

我们发现一个有趣的特征来连接这八个图。在一个空的停车场中,当第一辆车到达时,它最初停放在I/O单元格上。这是单车图的起始节点。这辆车可以移动到停车场内的不同位置。一旦车辆离开I/O单元格,停车场就可以接收第二辆车。当第二辆车到达时,它将自己停放在I/O单元格上。这是2车图的起始点。之后,车辆可以重新定位到停车场内的其他位置。类似地,当这两辆车都停在I/O单元格之外的任何位置时,停车场就可以接收第三辆车,我们可以一直进行此过程直到8辆车。因此,我们观察到每个$k$车图($k = 1, 2, \dots, 8$)都从I/O单元格开始。因此,我们将包括占用的I/O单元格的停车场配置定义为初始节点。2车图的初始节点在图\ref{f4-zh}b中以红色突出显示。请注意,每个$k$车图都有多个初始节点,除了1车图。为了将此节点与其他初始节点区分开,我们将其定义为根节点(见图\ref{f4-zh}a中的大红色节点)。我们还观察到为了容纳新车辆进入停车场,我们必须确保没有车辆停在I/O单元格上。这样的配置——I/O单元格未被占用——始终允许新车辆进入停车场。我们将这些配置定义为开放节点(见图\ref{f4-zh}a中的绿色节点)。如果I/O单元格中的任何一个被占用,新车辆将无法进入停车场,这些节点称为非开放节点。在图\ref{f4-zh}a中有五个非开放节点(灰色节点):(11, 21)、(11, 12)、(21, 22)、(21, 31)和(31, 41)。尽管(31, 41)不会立即阻塞停车场,但它只允许第二辆车停在(11, 21)上,之后不能再有车辆进入停车场。

\fig{f5}{0.7}{$4 \times 4$停车场的状态空间图。此图具有5,913个顶点和14,635条边。为了可视化目的,已注释其中一些顶点。}

当第二辆车到达I/O点时,它成为2车图的初始节点。因此,1车图的开放节点和2车图的初始节点通过一种进入车辆的操纵连接在一起。同样,2车图的开放节点和3车图的初始节点也是如此。我们称这些连接器为桥接边缘。1车和2车图的桥接边缘在图\ref{f4-zh}c中用蓝色突出显示。我们一般化,k车图的开放节点和$(k + 1)$-车图的初始节点使用桥接边缘连接。有了桥接边缘,我们就有了一个代表整个状态空间的图(请参见图\ref{f5-zh},显示了$4 \times 4$停车场的状态空间)。

\fig{f6}{1}{基于谜题的检索测试。(a) 一个带有3辆车的$4 \times 4$停车场。(b) 我们在左侧的布局上执行基于谜题的检索测试,并发现通过重新定位停车场内的其他车辆,可以检索出所有三辆车。}

该状态空间图具有5,913个顶点和14,635条边。在此图中,有72个连接组件,即彼此可达的顶点集。来自一个连接组件的顶点与来自其他连接组件的顶点之间不进行通信。

然而,连接组件内的所有顶点都可以相互到达。为了更好地可视化图,我们已注释了一些节点。例如,图\ref{f5-zh}a是一个只有一辆车的停车场,图\ref{f5-zh}d是一个有四辆车的停车场,依此类推。为了确定有多少辆车可以驶入停车场,我们从寻找1车图的顶点开始。1车图始于第一辆车进入停车场的时刻,即根节点。如果我们探索根节点的连接组件,我们可以达到高达7车图的顶点。其余的连接组件不是从根节点开始的。因此,不可能通过驾驶车辆达到这些配置。例如,图\ref{f5-zh}i、图\ref{f5-zh}j、图\ref{f5-zh}k和图\ref{f5-zh}l中的顶点不能从根节点到达。因此,在$4 \times 4$停车场中最多可以停放7辆车。图\ref{f5-zh}g显示了一个有7辆车的停车场。总之,尽管$4 \times 4$停车场有容纳8辆车的能力,但我们只能驾驶7辆车进入。查找我们可以驾驶进入$m \times n$停车场的最大车辆数的步骤如下:

\begin{itemize}
    \item 计算可以放置在停车场中的最大车辆数,$k_{max} = \lfloor\frac{m\times n}{2}\rfloor$。
    \item 为具有$k$辆车的停车场生成状态空间图,其中$k = 1, 2, \dots, k_{max}$。
    \item 对于每个图,识别初始节点和开放节点。
    \item 将$k$车图的开放节点与$(k +1)$车图的初始节点连接起来。这将产生一个具有一组连接组件的图。
    \item 找到带有根节点的连接组件。该组件由1车图到$k_{le}$车图的顶点组成,其中$k_{le}\le k_{max}$。
    \item 返回$k_{le}$的值,即可以驾驶进入停车场的最大车辆数。
\end{itemize}

我们已经了解到,在$4 \times 4$停车场中,我们最多可以停放7辆车。我们只能以后进先出的顺序检索这7辆车。然而,我们并不总是希望为了让另一辆车离开而离开停车场。因此,我们想知道:在停车场中最多可以停放多少辆车,以便其中的任何一辆车在其他车辆保留的情况下离开?为了回答这个问题,我们选择一个有$k$辆车的停车场,其中$k = 1,2,\dots,7$,并测试是否有可能让任何一辆车离开。由于这个测试类似于Rush Hour拼图,我们称之为基于谜题的检索测试。例如,在图\ref{f6-zh}中,我们对一个有3辆车的停车场进行了基于谜题的检索测试。我们观察到在所有三种情况下,目标车辆达到了I/O点,而其余车辆不需要离开停车场,而是重新定位以使目标车辆离开。然后,我们对整个状态空间进行基于谜题的检索测试,以找到可以停放在$4 \times 4$停车场中的车辆数。然而,为了最小化我们的搜索工作量,我们从具有7辆车的停车场配置开始这个测试。我们发现这样停放7辆车是不可能的。然后我们探索具有6辆车的配置,测试再次失败。最后,对于具有5辆车的配置,我们发现所有车辆都可以离开。我们不需要对具有少于5辆车的配置继续进行这个测试,因为肯定会有一些配置通过这个测试。因此,我们发现在$4 \times 4$停车场中,最多可以停放5辆车,以便其中的任何一辆车在其他车辆保留的情况下离开。

在传统设置中,我们希望停放的方式是任何一辆车都可以离开,而不需要重新定位其他车辆。因此,我们想知道:在停车场中最多可以停放多少辆车,以便没有车辆被阻挡?为了回答这个问题,我们选择一个有$k$辆车的停车场,其中$k = 1,2,\dots,5$,并测试是否有可能让任何一辆车离开停车场,而不需要重新定位其他车辆。我们将这个测试称为未阻挡检索测试。然后,通过进行蛮力操作,我们发现在$4 \times 4$停车场中,最多可以停放4辆车,以便没有车辆相互阻挡。

\fig{f7}{1}{三种停车场设计。}
\fig{f8}{1}{$A^*$搜索的单车启发式。为了从左侧停车场中检索深色车辆,右侧停车场被用作启发式,只在与深色车辆相同的位置放置一辆车。(有关本图例中颜色的解释,请参阅本文的网络版本。)}

观察状态空间图,我们发现存在三种停车场设计。第一种类型是停车场,最多停放车辆,车辆完全相互阻挡,车辆只能按照后进先出的顺序离开。我们将这种设计称为有限出口(图\ref{f7-zh}a)。这种停车场通常在不同的活动中发现,如音乐会或球赛。然后是第二种类型,车辆也相互阻挡,但任何车辆都可以通过移动其他车辆离开,就像Rush Hour拼图一样。我们将这种设计称为完全出口(图\ref{f7-zh}b)。最后,第三种是传统停车场(图\ref{f7-zh}c),在这种停车场中,没有车辆相互阻挡。我们发现了这些设计,因为我们在提出研究问题时没有预设任何特定的布局或结构。

\subsection{检索车辆需要多长时间?}

每辆车的检索时间是这些设计的性能指标之一。我们使用相同的状态空间图来计算检索时间。我们在状态空间图上应用$A^*$搜索算法来计算停车场中每辆车的最佳检索。

$A^*$是一种启发式搜索算法,通过扩展根据最小路径成本选择的最有前途的节点来探索图。路径成本评估为$f(s) = g(s) + h(s)$,其中$g(s)$是从初始状态到当前状态s的成本,$h(s)$是从当前状态s到目标状态的最便宜路径的估计成本。Hart等人还表明,如果启发函数$h(s)$是可接受的和一致的,则$A^*$算法保证产生最优结果。一个可接受的启发式$h(s)$是这样构建的,以便它生成到达目标状态的实际成本$h(s^*)$的下界:$h(s)\le h(s^*)$。一致的启发式是这样的,对于每个状态s和s的每个后继状态$s'$,从s到目标的估计成本$h(s)$不大于到$s'$的步骤成本$c(s, s')$加上从$s'$到目标的估计成本$h(s')$,可以表达为$h(s)\le c(s, s') + h(s')$。一致的启发式必然是可接受的,而反之则未必成立。我们已经简要介绍了$A^*$搜索的文献,更多信息请参阅Hart等人,Pearl,Russell和Norvig。

在我们的$A^*$算法实现中,输入包括:起始节点、目标车辆和启发式函数。起始节点是初始状态,由停车场的大小、停放的车辆数量和它们的位置表示。目标车辆是我们要检索的车辆。作为启发式函数,我们使用单车检索时间。例如,在图\ref{f8-zh}中,为了估算从左侧停车场检索深色车辆的成本,我们在右侧停车场的与目标车辆相同的位置放置一辆车,并计算检索时间。深色车辆的检索成本为27(图\ref{f8-zh}a),启发成本为5(图\ref{f8-zh}b)。由于这是我们问题的简化版本,它始终是一致的,因此我们可以得到最优结果。我们将这个启发式函数称为单车启发式。

我们的检索算法通过使用两个列表来探索状态空间:i) frontier(前沿)和ii) explored(已探索)。前沿列表是一个按照$f(s)$值对所有待访问节点进行排名的优先级队列,而已探索列表则存储所有已访问的节点。在开始时,前沿列表只包含初始节点,当选择访问目标节点时搜索停止。目标节点是目标车辆位于I/O点的配置。我们还创建了一个称为父-子的第三个列表,用于跟踪节点之间的关系——每当我们的搜索达到目标节点时,此列表用于构建从初始节点到目标节点的路径。应用$A^*$搜索,我们已经计算出从图\ref{f9-zh}中的有限出口、完全出口和传统停车场检索车辆所需的最佳移动次数。请注意,在有限出口停车场中,车辆的检索遵循后进先出的顺序,我们首先检索第7辆车,然后是第6辆车,依此类推(见图\ref{f7-zh}a),然后计算累积检索时间,如图\ref{f9-zh}a所示。

\fig{f9}{1}{车辆检索时间。}

在这里,我们计算检索时间,假设车辆在一个单元格内行驶需要一个时间单位。然而,读者可能对在真实停车场中的检索时间感兴趣。在这方面,我们在这里展示一个例子。一般而言,一个停车位的尺寸是$9' \times 18'$。由于我们考虑的是一个$1 \times 2$的停车位,我们可以将一个单元格的尺寸视为$9' \times 9'$。根据Schwesinger等人的说法,停车场中自动驾驶车辆的速度限制为$10 km/h (≈ 9ft/s)$。因此,一个单元格的行驶时间为1秒。在完全出口的停车场(图\ref{f9-zh}b)中,为了取回车辆02,所有车辆共行驶27个单元格。因此,车辆02的检索时间为27秒。


\entitle
\phantomsection
\addcontentsline{toc}{chapter}{原文2}
\noindent \zihao{4}\heiti 原文2\newline
\tnr

\begin{center}
    \zihao{4}\textbf{Spatial planning of urban communities via deep reinforcement learning}\\    
    \zuozhe Yu Zheng, Yuming Lin, Liang Zhao, Tinghai Wu, Depeng Jin,Yong Li\\
    \xuexiao Tsinghua University, Beijing, P. R. China.\\
    liyong07@tsinghua.edu.cn\\
\end{center}

\fontsize{10.5pt}{10.5pt}\selectfont

\noindent\textbf{Abstract: }Efective spatial planning of urban communities plays a critical role in the sustainable development of cities. Despite the convenience brought by geographic information systems and \
computer-aided design, determining the layout of land use and roads still heavily relies on human experts. Here we propose an artifcial intelligence urban-planning model to generate spatial plans for urban \
communities. To overcome the difculty of diverse and irregular urban geography, we construct a graph to describe the topology of cities in arbitrary forms and formulate urban planning as a sequential \
decision-making problem on the graph. To tackle the challenge of the vast solution space, we develop a reinforcement learning model based on graph neural networks. Experiments on both synthetic and real-world \
communities demonstrate that our computational model outperforms plans designed by human experts in objective metrics and that it can generate spatial plans responding to diferent circumstances and needs. We \
also propose a human–artifcial intelligence collaborative workfow of urban planning, in which human designers can substantially beneft from our model to be more productive, generating more efcient spatial plans \
with much less time. Our method demonstrates the great potential of computational urban planning and paves the way for more explorations in leveraging computational methodologies to solve challenging real-world \
problems in urban science.

\section{Methods}
\subsection{Problem formulation}
We formulate the problem of community spatial planning as a sequential Markov decision process (MDP), an interactive process between \
the planning agent and the environment, in which the agent observes \
the ‘state’ (the current conditions of the community), takes an ‘action’ \
(placement of an urban functionality) at each step and receives ‘reward’ \
(effect of the planned result) signaled by the environment, which \
undergoes a ‘transition’ (changes of the layout) according to the agent’s \
action. We utilize DRL to learn an effective policy that maps states to \
actions with a parameterized neural network. The neural network is \
optimized towards higher spatial efficiency through massive training \
under the MDP, with millions of training samples of the 4-tuple (state, \
action, reward and transition). As illustrated in Extended Data Fig. 1, \
our MDP is composed of two consecutive stages:

\begin{itemize}
    \item Land-use planning. Given the initial road conditions, the agent 
    places functionality blocks one at a time, either near existing 
    roads or near boundaries of previously placed land use. After 
    all the functionalities and open spaces are allocated, a reward 
    regarding the efciency of land use is returned to the agent, 
    which treats diferent land use as an integrated system. The fnal 
    land-use plan becomes the initial condition of road planning.
    \item Road planning. Boundaries of planned land use are viable locations for road construction. The agent builds roads iteratively, 
    turning one boundary into a road segment at a single step. 
    Stopped at a predefned termination step, a reward considering 
    the transportation efciency is returned to the agent.
\end{itemize}

The reward is only calculated at the last step of each stage to \
summarize the performance of land-use planning and road planning, \
respectively, and all intermediate steps receive a reward of 0. We define \
the reward for land use and road layout based on the 15-minute-city \
concept\
, which emphasizes the spatial efficiency to facilitate active \
transportation such as walking and cycling instead of automobiles. \
The two reward terms are calculated as follows:
\begin{equation}
    r_L = \alpha Service +Ecology,\label{land-use-planning}
\end{equation}
\begin{equation}
    r_R = Traffic,\label{road-planning}
\end{equation}
where Service measures the community life-circle index in the 15-minute city, Ecology measures the coverage of green space and parks, \
Traffic is a combination of road density and connectivity (Methods) \
and α serves as a hyper-parameter indicating the weight of service \
performance in the land-use reward. With the calculated reward values, \
we use proximal policy optimization to update the parameters of value \
and policy networks. We first train the agent for the land-use planning \
task until convergence and then train the agent to build roads with the \
optimal land-use plan obtained in the first stage. After two stages of \
training, the AI agent is able to design communities with an efficient \
spatial layout of both land use and roads.

\textbf{Graph model.}Different from previous Go and chip design tasks, \
urban planning is more challenging because of its much larger degrees \
of freedom in the problem form. Specifically, the conditions of previous tasks are regular, for example, placing stones on a 19 × 19 board or \
placing rectangular macros onto a grid chip canvas, which can be represented by pixels (raster). By contrast, the conditions for community \
spatial planning are diverse and irregular because the road corners and \
land blocks are usually not orthogonal. To accurately describe urban \
geographic elements, including land blocks (L), segments of roads and \
land-use boundaries (S) and junctions between roads and land-use \
boundaries (J), we use vector representations, which have been proven \
to have substantial advantages over raster representations in urban \
planning, and consist of the following three geometries:
\begin{itemize}
    \item  ‘Polygon’ that describes a vacant land to be planned (for \
    example, L1 in the right part of Extended Data Fig. 2a) or an \
    already planned land block (for example, L2 in the right part of \
    Extended Data Fig. 2a) with the coordinates of the land boundary;
    \item ‘LineString’ that represents a road segment (for example, S3 in 
    the right part of Extended Data Fig. 2a) or a boundary edge of a 
    land block (for example, S9 in the right part of Extended Data 
    Fig. 2a) with the coordinates of the start and end points; and
    \item ‘Point’ that stands for the junctions between roads and landblock boundaries (for example, J2 and J7 in the right part of \
    Extended Data Fig. 2a) with their coordinates.
\end{itemize}

We transform all geographic elements into the above three categories of geometries and then represent the whole community as a \
graph, in which nodes are the geometries and edges stand for the spatial \
contiguity relationship between these geometries, that is, two nodes \
are connected if the underlying two geometries touch each other. Each \
node stores its geographic information as the node features, including \
the type, coordinates, width, height, length and area of the geometry. \
In this way, spatial planning can be transformed as a problem of making \
choices on a dynamic graph (Extended Data Fig. 2), in which the graph \
evolves according to the agent’s actions.

In the land-use planning task, the agent selects one L–J edge that \
connects a vacant land and a junction, placing a given functionality at the \
location specified by the corresponding L and J (Extended Data Fig. 2a). \
In each step, the topology of the contiguity graph changes because \
the newly placed functionality generates new nodes and edges. New \
nodes include the new functionality itself, its boundaries, new junctions \
and split segments. New edges indicate the newly established spatial \
contiguity. Similarly, in the road planning task, the agent selects one \
S node that is currently a boundary and constructs it as a road segment \
(Extended Data Fig. 2b). Although the topology remains the same, \
the graph’s attributes alter because the selected node’s type changes \
from boundary to road. Through the problem reformulation with the \
graph model, we can now handle the irregular urban blocks and unify \
the two seemingly distinct stages of land use and road planning on \
one single graph.

\textbf{Action space design.}Another major challenge of urban planning is \
the huge action space, which is almost infinite in the original continuous space, and still too large in the reduced discrete graph space. The \
contiguity graph continues to grow as we place a functionality at each \
step, resulting in a large graph with thousands of nodes and edges. \
A typical spatial plan of a 2 km by 2 km community can take a total \
number of 100 planning steps in each stage, and the contiguity graph \
can have 4,000 edges and 1,000 nodes, which makes the action space \
$4,000^{100}$ and $1,000^{100}$ for the two stages, respectively. In addition, valid \
actions are extremely sparse in the space, and a substantial portion \
of actions is of low quality and will lead to unreasonable results, such \
as placing a facility in the center of a vacant land without connecting \
roads. Therefore, it is crucial to reduce the action space and avoid \
unreasonable actions.

To address this challenge, we propose a general DRL framework in \
which an intelligent agent perceives and makes decisions in a reduced \
graph space, and the environment handles urban elements in the original geographic space and generates graph states according to the geographic spatial layout. Meanwhile, we decompose the entire action \
space into a Cartesian product of three sub-spaces, including what \
to plan, where to plan and how to plan, and let the DRL agent focus on \
the core issue of where to plan. The first sub-space of what to plan can \
be eliminated by fixing the planning order of different land-use types \
through domain knowledge, allowing land-use types that are more \
dependent on the initial road network to be planned earlier (Methods). \
To avoid apparently improper actions in where to plan, we impose \
planning constraints on the agent’s actions, with an action mask that \
blocks out unreasonable options, that is, only L–J edges and S nodes are \
candidates for the two planning stages, respectively. After selecting one \
L–J edge for a given land-use type, the functionality is placed in the corresponding land block (L node) at the location of the corresponding junction ( J node), whose shape and size are determined by predefined rules \
that maximize the reuse of existing roads and boundaries (Methods); \
thus, the last sub-space of how to plan is effectively eliminated. Through \
these designs, we narrow the action space to a solvable scale and filter \
out most unreasonable actions, enabling efficient optimization for \
DRL algorithms. In summary, the original problem of spatial planning \
is successfully transformed into a standard sequential decision-making \
process on a graph with moderate action space.

\textbf{Framework.}After the above problem reformulation and action space \
design, we propose a DRL framework in which an AI agent learns to \
lay out land use and roads by interacting with the spatial planning \
environment, as illustrated in Extended Data Fig. 3. The sequential \
MDP (Extended Data Fig. 3e,f) contains the following key components:
\begin{itemize}
    \item States summarize the current spatial plan with the previously 
    introduced contiguity graph containing rich node features, and 
    other information, such as statistics of diferent land use types.
    \item Actions indicate the locations to place the current land use or 
    construct a new road segment, which are transformed from the 
    selected edges or nodes in the contiguity graph.
    \item Rewards are 0 for all intermediate steps, except for the last step 
    in each stage, in which it evaluates the spatial efciency of land 
    use and roads.
    \item Transitions describe the changes of the layout given the \
    selected location, and the transitions occur in both the original \
    geographic space (new land use and road on the map) and the \
    transformed graph space (new topology and attributes of the \
    graph).
\end{itemize}

At each step, the agent represents the state by encoding the graph \
with a GNN. Via multiple message passing and non-linear activation \
layers, the GNN state encoder generates effective representations of \
edges, nodes and the whole graph (Extended Data Fig. 3a), which will be \
leveraged by the value and policy networks (Extended Data Fig. 3b–d). \
Specifically, because choosing locations for land use is equivalent to \
selecting edges on the graph, the land-use policy network takes the \
edge embeddings and scores each edge with an edge-ranking MLP, \
as shown in Extended Data Fig. 3b. The obtained score for each edge \
indicates the sampling probability of the corresponding edge, which \
is returned to the environment and becomes the probability of placing \
the land use at the location specified by that edge. Similarly, in road \
planning, the road policy network takes node embeddings and scores \
each node with a node-ranking MLP (Extended Data Fig. 3d), outputting \
the probability of choosing one land block boundary and building it as \
a road segment. Finally, the value network takes in the graph embedding that summarizes the whole community and predicts the planning \
rewards with a fully connected layer (Extended Data Fig. 3c). To master \
the skills of spatial planning, millions of spatial plans are accomplished \
by the proposed model to search the large solution space during the \
training process, which is utilized as real-time training data to update \
the parameters of the neural network.\\
\subsection{Detailed methodology}

\textbf{Framework.}As introduced in the paper, we use vector geometries \
including Polygon, LineString and Point to describe urban geographic \
elements. Specifically, there are ten types of land blocks that are represented as Polygon, including the initial vacant land to be planned, and \
nine different functionality types, which are residential (RZ), school \
(SC), hospital (HO), clinic (CL), business (BU), office (OF), recreation \
(RE), park (PA) and open space (OP). In addition, there are two types \
of segments (roads and land-use boundaries) that are represented by \
LineString and one type of junction (intersections between roads and \
land-use boundaries) that is represented by Point. Therefore, a community is faithfully represented by a table of geometries, in which each row \
is a geographic element with three columns of ID, type and geometry. \
Initial conditions of a community consist of all the original land blocks, \
roads and intersections, whose accurate coordinates are recorded \
by their corresponding geometries in the table of geometries. In the \
synthetic grid community, we experiment on a basic community with \
a size of 2.4 km by 2.4 km, containing 16 rectangular vacant lands, 40 \
horizontal or vertical initial road segments and 25 road intersections, \
as shown in the first step of Extended Data Fig. 1. In the real-world community, we replicate the road network of HLG and DHM communities in \
Beijing from OpenStreetMap using OSMnx30 and geopandas, reserve \
residential blocks and leave other areas as vacant land to be renovated. \
Finally, we obtain two communities of around 4 km\textsuperscript{2} as shown in Fig. 1a\
and Supplementary Fig. 11a.

\textbf{Planning needs and requirements.}Before carrying out the actual \
spatial planning, we need to determine the planning needs and requirements, which serve as the configuration of the planning environment. \
The planning need describes the amount that each land-use type has\
to achieve, either in area or in number, for example, residential blocks \
of 50\% community area and three hospitals. Meanwhile, we also have \
requirements on the minimum area (in square meters) of each planned \
block; for example, the area of one school is at least 10,000 m\textsuperscript{2}\
. Supplementary Table 3 shows an example of the planning needs and requirements for a community, in which 15\% of the community area needs to \
be planned as parks; thus, it serves as a green community. Only spatial \
plans that satisfy all the needs and requirements are considered as \
successful episodes and reserved as training samples, and those failed \
episodes are discarded. In our framework, the planning needs and \
requirements are configurations for the environment, making our \
model highly flexible in generating spatial plans. Specifically, once \
we obtain a well-trained model under one configuration, we can simply change the configuration and directly perform model inference \
without re-training to generate plans for different planning needs \
and requirements, such as the community plans of different service \
supplies in Fig. 1c,d.

\textbf{Planning order of land-use types.}As introduced in the paper, in order \
to reduce the huge action space, we fix the planning order of different \
land-use types based on domain knowledge and make the agent focus \
on the core task of selecting locations. Because feasible locations \
are next to existing junctions, land use that is planned earlier will be \
closer to initial roads with more convenient traffic. Therefore, we first \
plan those facilities that depend more on roads, including hospitals \
(clinics), schools and recreation. Meanwhile, at later steps of land-use \
planning, the shape of feasible vacant lands tends to be more irregular \
and fragmented, which is not suitable for residential blocks that usually \
occupy a whole plot of land; thus, we distribute residential blocks after \
planning the above road-dependent facilities. Finally, we arrange those \
land-use types that are not much demanding in land shapes. After all the \
planning needs are satisfied, the remaining vacant lands are assigned \
as open spaces. In summary, the planning order in our framework is \
fixed as follows: hospital, school, clinic, recreation, residential, park, \
office, business and open space. Letting the agent determine the order \
of land-use types may be an alternative approach. However, it will make \
the problem much more complicated, as the action space is increased \
drastically. In practice, our fixed order generates sound spatial plans.

\textbf{Land cutting.}In land-use planning, the environment receives the \
action from the agent, which is the selected L–J edge, and cuts a new \
land from the corresponding land block (L node) at the location of the \
corresponding junction ( J node). We develop a rule-based system with \
expert knowledge incorporated to determine the shape and size of the \
new land. The rule-based system is roughly composed of three steps: \
(1) Determine the relationship between J and L, such as in the middle of \
a road or at the corner. (2) Determine the reference line along existing \
boundaries from junction J, which can be I-shape, L-shape and U-shape. \
(3) Determine the length of inward extension from the reference line \
into the block L, forming the final sliced new land. The three steps \
are conducted according to expert knowledge, in order to meet the \
planning requirements and fit the current plan as closely as possible.

\textbf{State.}Our state contains three parts: (1) urban contiguity graph, \
(2) current object to be placed and (3) community statistics. We construct \
a graph to represent the current community information as illustrated \
in Extended Data Fig. 2, in which nodes are urban geographic elements \
and edges indicate the spatial contiguity relationship. We compute \
rich geographic attributes as node features, including the type, coordinates, area, length, width and height of the underlying urban element. \
The edges are represented by a sparse adjacency matrix. As for the \
current object to be placed, its type is determined by the environment \
according to planning needs and planning order, that is, the environment will traverse the planning order and transit to the next type if the \
planning needs for the previous type have been satisfied. We treat the \
current object as a virtual isolated node, with its type feature provided \
by the environment and other node features left as default values. \
Lastly, community statistics include the area and count of different \
land-use types in the current plan, as well as the planning needs, which \
summarize the current conditions and the progress of spatial planning.

\textbf{Action.}As illustrated in Extended Data Fig. 2a, land-use planning is \
reformulated as a sequential MDP in which the agent selects an edge in \
a dynamic graph. Therefore, the action space for land-use planning is \
the probability distribution of choosing from N edges, and we sample \
from this distribution to obtain the action. Similarly, road planning is a \
sequential MDP of choosing nodes as shown in Extended Data Fig. 2b; \
thus, the action space for road planning is the probability distribution \
over M nodes, which is sampled to generate the node selection action. \
In addition, as introduced previously, we impose constraints on the \
action space; for example, the agent can only select L–J edges (between \
vacant lands and junctions) and S nodes (land-use boundaries) to \
avoid unreasonable spatial plans. Thus, we calculate a mask in each \
step that indicates feasible options, and the probability distribution \
will be multiplied by the mask, allowing only feasible edges or nodes \
to be sampled as actions.

\textbf{Policy and value networks.}As shown in Extended Data Fig. 3b–d, \
we develop separate policy networks to take actions in the policy \
and value stages, respectively, as well as a value network to predict \
the performance of spatial plans. The three networks share the same \
state encoder to obtain state representations, taking full advantage \
of the GNN. Policy networks generate the probability distribution by \
scoring the edges and nodes in the graph, and then sample from this \
distribution to take actions. Meanwhile, the value network evaluates \
the whole graph to predict spatial efficiency, providing feedback for \
the community plan. In this section, we introduce the detailed design \
of the three networks.

\textbf{Land-use policy network.}In land-use planning, the agent places the \
current object at the location specified by the selected edge. The effect \
of edge selection is related to both the edge and the current object; for \
example, placing a hospital next to an already planned hospital may \
lead to low service efficiency. Therefore, the land-use policy network \
considers both the edge and the current object as input. As shown \
in Extended Data Fig. 3b, a feed-forward network, which is an edgeranking MLP, is developed to score each edge:
\begin{equation}
    s(e_{ij})=FF_{land}(e_{ij}^L||v_c||e_{ij}^L-v_c||e_{ij}^L·v_c),\label{each-edge-score}
\end{equation}
here the difference and the inner product of $e^L_{ij}$ and $v_c$ are also concatenated to emphasize the relationship between the current object to 
be planned and those already planned land use. The scores are converted to a probability distribution over all edges using softmax:
\begin{equation}
    Prob(e_{ij})=\frac{e^{S(e_{ij})}}{\sum_{s,t\in E}e^{s(e_{st})}},\label{scores-converted-to-probability-distribution}
\end{equation}
which is sampled to select an edge.

\textbf{Road policy network.}In road planning, the agent selects one boundary \
node and plans a road at its location. Different from that of land-use \
planning, the topology of the graph is stable with no new nodes to be \
added. Thus, there is no need to include the current object. Meanwhile,\
the road policy network takes node embeddings as input, which already \
contain neighbor geographic information through message passing of \
GNN. As in Extended Data Fig. 3d, another node-ranking feed-forward \
MLP is adopted to score each node:
\begin{equation}
    s(v_i) = FF_road(v_i).\label{score-based-in-node-ranking-feed-forward-MLP}
\end{equation}
The score is also transformed to probability with a softmax operator:
\begin{equation}
    Prob(v_{ij})=\frac{e^{S(v_{ij})}}{\sum_{j\in N}e^{s(v_{j})}},\label{score-transformed-to-probability}
\end{equation}
and we sample from this probability distribution to select one node.

\textbf{Value network.}As shown in Extended Data Fig. 3c, we develop a value \
network to judge the current planning situations and predict the planning \
performance. Because it is an overall evaluation of the entire community, \
we take the graph-level representation as the input of the value \
network. Meanwhile, we also include community statistics. Specifically, \
we concatenate graph representations and the statistics embedding, \
and adopt a fully connected layer to predict the performance:
\begin{equation}
    \hat{v} = fa(g^L||h_s),\label{predict-performance}
\end{equation}
where $\hat{v}$ is the estimated value of the current plan.

\textbf{Reward.}We train the policy networks to optimize the efficiency of \
spatial layout, with respect to service, ecology and traffic. As in \
equations (1) and (2) of this paper, we define reward functions that give a \
comprehensive evaluation of the above metrics. Meanwhile, the reward \
values can be quickly computed within tens of milliseconds given a \
spatial plan, making it possible to collect large-scale samples for \
training DRL models. In this section, we introduce how the three metrics are \
calculated. It is worth noting that our framework is flexible and can be \
extended to include more metrics in the reward.

\textbf{Service.}We adopt the concept of 15-minute life circle, which requires \
that basic services of the community be reachable for residents within \
15 min by walking or cycling. Specifically, as demonstrated in Fig. 1b,c, \
we consider five different basic services, each of which is related to \
one or two facilities, that is, education (school), medical care \
(hospital, clinic), working (office), shopping (business) and entertainment \
(recreation). Therefore, the 15-minute life circle means that the distances \
between facilities and residential zones need to be less than the \
walking distance of 15 min, which is set as 500 m in our experiments. \
We define the service metric as the proportion of accessible services \
within 500 m, and the metric is averaged for all residential zones. \
Formally, given a community spatial plan p, the service metric is \
calculated as follows:
\begin{equation}
    d(i,j) = min\{EucDis(RZ_i,FA_1^j),\dots,EucDis(RZ_i,FA_{n_j}^j)\},\label{min-distance-for-the-ith-residential-zone}
\end{equation}
\begin{equation}
    Service_i = \frac{1}{5}\sum_{j=1}^{5}\mathbbm{1}[d(i,j)<500],\label{life-circle-metric-for-the-ith-residential-zone}
\end{equation}
\begin{equation}
    Service = \frac{1}{n_{RZ}}\sum_{i=1}^{n_{RZ}}Service_i,\label{life-circle-metric-for-all-residential-zones}
\end{equation}
where $EucDis$ is the Euclidean distance, $d(i,j)$ is the minimum distance \
for the $i$th residential zone $RZ_i$ to access the $j$th service that is provided \
by facility $FA^j$ and $n_j$ is the total number of facilities $FA^j$. $Service_i$ is the \
15-min life-circle metric for the ith residential zone, and we average over \
all $n_{RZ}$ residential zones to obtain the final service metric for the whole \
community. This service metric guides the agent to arrange facilities\
in a more decentralized way and close to residential zones, which is \
critical for increasing the ability of community services.

\textbf{Ecology.}The ecology of a community is important to the physical and \
mental health of residents; thus, we include an ecology metric that 
measures the layout efficiency of parks and open spaces. In general, 
parks and open spaces serve the residents who live in the neighborhood, \
and we hope they can serve as many residential areas as possible. \
Formally, we define the ecological serving range as the region within \
300 m from a park or an open space, and the ecology metric measures \
the proportion of residential areas that are covered by the ecological \
serving range. The metric is calculated as follows:
\begin{equation}
    \begin{aligned}
        ESR = & Union\{Buffer(PA_1,300),\dots,Buffer(PA_{n_{PA}},300),\\
        & Buffer(OS_1,300),\dots,Buffer(OS_{n_{OS}},300)\},\label{ecological-serving-range}
    \end{aligned}
\end{equation}
\begin{equation}
    A_{RZ} = \sum_{i=1}^{n_{RZ}}Area(RZ_i),\label{all-residential-areas}
\end{equation}
\begin{equation}
    A_{RZ}^e = \sum_{i=1}^{n_{RZ}}Area(Intersection(RZ_i,ESR)),\label{intersection-between-all-parks-and-open-spaces}
\end{equation}
\begin{equation}
    Ecology = \frac{A_{RZ}^e}{A_{RZ}},\label{ecology}
\end{equation}
where $Buffer(PA_i,300)$ and $Buffer(OS_i,300)$ represent the regions that \
extend the park and open space 300 m outward, which is their serving \
range, and $ESR$ is the ecological serving range that combines the serving \
range of all parks and open spaces. The ecology metric encourages the agent \
to maximize $A^e_{RZ}$; thus, the greenness of the community plan is promoted.

\textbf{Traffic.}For the second stage of road planning, we evaluate the traffic 
efficiency from three perspectives, including density, connectivity
and spacing. Road density is the ratio of the total length of roads to 
the land area. Connectivity is a network characteristic that reflects the 
strength of how different parts of a network are linked with each other, 
and we choose the number of connected components and the number 
of dead-end roads. To achieve appropriate road spacing, we also include 
two terms to penalize too large (>600 m) and too small (<100 m) spacing. 
Formally, the traffic metric is calculated as follows given the road 
plan $p_R$ and the converted graph $g_R$ from the planned road network:
\begin{equation}
    T_{density}=\frac{Length(p_{R})}{A_c},\label{density}
\end{equation}
\begin{equation}
    T_{connectivity}=\frac{1}{NCC(g_R)}+\frac{1}{1+\sum_{v \in g_R}\mathbbm 1[Degree(v)=1]'},\label{connectivity}
\end{equation}
\begin{equation}
    T_{spacing}=\frac{1}{1+\sum_{r \in p_R}\mathbbm 1[Length(v)>600]}+\frac{1}{1+\sum_{r \in p_R}\mathbbm 1[Length(v)<100]'},\label{spacing}
\end{equation}
\begin{equation}
    Traffic=\frac{1}{3}*(T_{density}+T_{connectivity}+T_{spacing}),\label{traffic}
\end{equation}

where Length calculates the length of a road segment, $A_c$ is the area \
of the community, $NCC$ calculates the number of connected components \
in a network and Degree calculates the degree of a node in the graph. \
Combining the three perspectives, the traffic metric encourages the \
agent to plan denser roads and, at the same time, guarantees \
connectivity and appropriate spacing, without creating dead-end roads \
or planning too-long or too-short road segments.

\textbf{Model training.}We train our model for hundreds of iterations \
to learn the skills of spatial planning. In each iteration, we collect \
training samples of a few thousand episodes and update the parameters of \
our model using proximal policy optimization. Specifically, the loss \
function is a combination of policy loss, policy entropy and value loss. \
Policy loss is a surrogate clipped objective to improve the policy with \
safe exploration, which is calculated as follows:
\begin{equation}
    r_t(\theta) = \frac{\pi_\theta(a_t|s_t)}{\pi_{\theta_{old}}(a_t|s_t)},\label{ratio-of-new-old-policy}
\end{equation}
\begin{equation}
    L_{policy} = min(r_t(\theta)\hat{A_t},clip(r_t(\theta),1-\epsilon,1+\epsilon)\hat{A_t}),\label{policy-loss}
\end{equation}
\begin{equation}
    \hat{A_t} = Q(s_t,a_t)-V(s_t),\label{advantage-function}
\end{equation}
where $\theta$ is the parameters of our model, $r_t(\theta)$ is the ratio of the \
probability of the new policy to the old policy, $\hat{A_t}$ is the advantage \
function and clip restricts the update to be not too large. The entropy loss controls \
the balance between exploitation and exploration, which is calculated as follows:
\begin{equation}
    L_{entropy} = Entropy[Prob(a_1),\dots,Prob(a_{n_a})],\label{entropy-loss}
\end{equation}
where $n_a$ is the total number of actions that equals to $M$(edges) or $N$
(nodes) in different planning stages, and Prob is obtained by policy 
networks according to equations (12) and (14). We use mean squared 
error loss to supervise the value prediction:
\begin{equation}
    L_{value} = MSE(\hat{v_t},R_t),\label{value-loss}
\end{equation}
where $R_t$ is the return value from ground-truth and $\hat{v_t}$ is estimated \
by value network according to equation (15). The final loss function is a \
weighted sum of the above three terms:
\begin{equation}
    L = L_{policy}+\beta L_{entropy}+\gamma L_{value},\label{sum-Loss}
\end{equation}
where $\beta$ and $\gamma$ are hyper-parameters in our model.

\textbf{Model inference.} After we obtain a well-trained model, we perform \
model inference to generate community plans. We use the policy \
networks and compute the probability distribution over different actions \
according to equations (12) and (14), that is, the probability of selecting \
different edges and nodes. Then the most likely action is chosen to place \
land use or road at the location specified by the action:
\begin{equation}
    a = \text{argmax}\{Prob(a1),\dots,Prob(a_{n_{a}})\},\label{action}
\end{equation}
where $n_{a}$ is $M$ or $N$ for land use and road planning, respectively. It is\ 
worth noting that we can directly perform model inference under a different \
setup without re-training, and the results are illustrated in Fig. 1c,d.

\textbf{Integration with manually designed planning concepts.} The DRL \
framework is not designed for replacing human designers but serves as an \
intelligent assistant to improve the productivity of human designers. \
Specifically, AI models are good at optimizing spatial efficiency in large \
solution spaces, whereas human designers are good at conceptual prototyping. \
Therefore, we design a new workflow, in which human and AI collaboratively \
accomplish urban-planning tasks and leverage their respective expertise. \
As shown in Supplementary Fig. 7a, we propose a workflow with four key \
steps of conceptualization,planning, adjustment and evaluation, where AI \
takes responsibility for the planning step. In the workflow, human designers \
can leave the heavy and specific planning work to the AI, and they only need \
to provide relatively abstract conceptual planning and make adjustments to \
the spatial plans generated by the AI. We represent planning concepts \
as two major types, center and axis, and each concept is related to one \
or several land-use functions. For example, in the left part of Supplementary \
Fig. 7b, the RE center in the HLG community represents the \
concept that encourages recreation zones near the specified location. \
Similarly, in the right part of Supplementary Fig. 7b, the BU\&OF axis in \
the DHM community represents the concept that expects a business \
and office core along the specified band region. We feed the initial \
conditions of the community and the planning concepts to the model, \
and then train our DRL model to realize the planning concepts while at \
the same time optimizing spatial efficiency.
As the concept of center and axis is essentially the spatial relationship \
between specific land-use functions and predefined locations, it \
can be easily integrated into our framework. Specifically, we utilize \
customized reward functions to implement planning concepts, that \
is, we add a reward to reflect the extent of consistency with the \
planning concept. For the center concept, we calculate the fraction of \
concept-related land-use functions in the region near the specified \
center location as follows:
\begin{equation}
    r_c = \frac{1}{n_c}\sum_{j=1}^{n_c}\mathbbm{1}[T_j\in T_c],\label{concept-related-land-use-functions}
\end{equation}
where $L_a$ is the length of the axis, $L^p_a$ is the distance of projected points \
of concept-related land-use functions on the axis, $n_a$ is the number of \
land use blocks within 100m from the predefined axis and $T_a$ is the \
land-use function related to the concept. This reward encourages the \
DRL agent to place concept-related land-use functions as evenly as \
possible in the band area around the axis. The concept reward is \
combined with efficiency reward, including service and ecology, by \
weighted sum. Through jointly optimizing efficiency and concept \
rewards, the DRL agent learns to improve spatial efficiency on the basis \
of realizing predefined planning concepts.

\zhtitle
\phantomsection
\addcontentsline{toc}{chapter}{译文2}
\noindent \zihao{4}\heiti 译文2\newline
\renewcommand{\fig}[3]{
    \begin{figure}[!htb]
        \centering
        \includegraphics[width=#2\textwidth]{#1.png}
        \caption{#3}
        \label{#1-zh}
    \end{figure}
}

\zihao{-4}\songti
\noindent 作者:Parag J. Siddique, Kevin R. Gue, John S. Usher\\
\noindent 出处:Transportation Research Part C: Emerging Technologies, Volume 127, June 2021, 103112
\vspace{2ex}

\begin{center}
    \zihao{4}\textbf{基于谜题的停车}
\end{center}


\noindent\textbf{摘要:} 在普通停车场中,大部分空间并非用于停车,而是用于车辆在停车位之间行驶的车道。车道不能被阻塞,一个简单的原因是:被阻塞的车可能需要在挡住它的车之前离开。自动停车和自动驾驶车辆的智能通信能力为克服这一限制提供了机会,因此可以实现更高的汽车存储容量。我们展示了如何通过集中控制器将干扰的车辆移开,从而最大化停车场中的汽车数量。我们提供了单入口小型停车场的最优结果,并为更大停车场提供了启发式方法。停车容量的提升可达80\%。

\section{基于谜题的停车场}
自主乘用车辆的前景为改善城市交通提供了许多有趣的机会。在停车方面,远程控制汽车的潜力允许汽车停放得更近,因为不需要打开车门让乘客下车。停车密度的提升潜力估计为20\%。另一个想法是,自动驾驶车辆不需要停在繁忙的城市区域。相反,可以在廉价的郊区创建远程停车区域。另一个被广泛接受的观点是,自动驾驶车辆将消除私人车辆所有权。自动出租车将带来共享出行,从而大幅减少停车需求。

在这一背景下,博世与梅赛德斯和福特合作,设计了一座为自动驾驶车辆设计的自动泊车场。到达停车场时,乘客在一个送车点离开车辆。停车场配备了智能传感系统,引导车辆找到停车位。每当乘客需要车时,可以通过智能手机应用程序检索车辆,车辆将到达取车点。值得注意的是,不需要人工控制汽车,也无需建设额外的市政基础设施。

在这样的自动化停车场中,通过消除行车道,可以实现更高的停车密度。在普通停车场中,行车道不能被封锁的一个简单原因是:被封锁的车可能需要在挡住它的车之前离开。然而,自动泊车场为克服这个问题,实现更高停车密度提供了机会。因此,我们已经拥有实现更高密度所需的技术,一切需要高效的设施设计和规划。

\fig{f1}{0.8}{其他作者提出的高密度停车设计。 (a) Ferreira等人 (2014) 提出了一种通过在平行车道上相互阻挡的设计来停放汽车。他们在设施的入口和出口都分配了缓冲区域,以便车辆可以进行必要的移动进行重新定位。 (b) Timpner等人在被阻挡的车道之间放置了垂直过道,以减少干扰车辆的数量。阻挡的车辆重新定位到过道上,为即将离开的车辆清出道路。 (c) Banzhaf等人通过添加一个车辆阻挡的过道来修改现有的停车场。 (d) 在Nourinejad等人的设计中,过道的宽度随着车道的深度而变化,以帮助容纳重新定位的车辆。}

一些作者设想了一些停车方案,以提高停放汽车的密度,或等效地提高停车场的容量。Ferreira等人首次提出了自主车辆的高密度停车方案。在他们的设计中,汽车通过在平行车道上相互阻挡来停放,类似于材料处理系统中使用的深车道存储系统(图\ref{f1-zh}a)。他们假设有一个自主停车控制器,负责控制停车场内的车辆移动。这一组研究人员已将这项工作扩展到考虑各种操作参数,如每辆车的停车空间、操纵次数、行驶距离和检索时间。他们的设计将停车容量提高了50\%。

为了减少干扰车辆的数量,Timpner等人修改了Ferreira等人的设计,减小了车道深度,并允许车辆从车道的任一端离开。作者估计密度提高了33\%。Banzhaf等人建议通过在现有过道内允许垂直的阻挡车道来修改现有布局。他们报告密度增加了25\%。Nourinejad等人描述了一个类似的系统,其中过道的宽度随着车道的深度变化,以帮助容纳重新定位的车辆。作者制定了一个混合整数非线性规划来最小化预期的车辆重新定位次数。然而,所有这些研究人员都计划针对更适合郊区地区的大型停车场。

从文献中我们了解到,自主车辆不会在停车场闲置。在放下乘客后,它将用于运送另一名乘客。然而,由于供需之间存在差距,当没有行程时会有一些空闲时间,因此车辆必须停在某个地方。如果这些车辆去郊区停车,它们将无法响应客户的呼叫。因此,自主车辆不会选择去城外停车,而是会在城市中心周围巡游以打发时间,从而导致城市拥堵。因此,我们无法完全消除城市停车。相反,我们可以寻找使停车更容易访问和更便宜的方法。为此,我们可以为城市区域提供小型而密集的停车场。这些停车场可以与零售业中的微型履行中心相类比。微型履行中心是靠近客户的小型仓库,专为按需电子商务履行而设计。每当自主车辆处于空闲状态时,它可以停在这些小型停车场。由于这些停车场将位于城市区域,因此每当车辆接收到行程请求时,它可以迅速服务请求。作为增值服务:在这些停车场中还可以引入车辆充电、清洁或类似的服务。

在本文中,我们研究了自主车辆停车场的高密度布局设计。我们的工作在两个重要方面与现有文献有所不同:

\begin{enumerate}
    \item 我们为小型停车场开发最优设计,这是人们可能在城市环境中找到的。
    \item 所有现有的研究都是基于一个布局,然后研究在该布局下的最优参数和性能。我们提出并回答了更基本的问题,即在不假设任何特定布局或结构的情况下,停车场可以停放多少辆车。正如我们将展示的,这种放宽导致了具有最大密度的设计,而不采用传统的“排和过道”停车结构。
\end{enumerate}

\fig{f2}{0.5}{Rush Hour难题。游戏的目标是通过移动其他车辆让红车直接移动到对面的出口。}

\fig{f3}{0.9}{(a) 一个$4 \times 4$停车场的示例,其中有两辆车,一辆水平停放在单元格(43, 44),另一辆垂直停放在单元格(23, 34),而单元格(11, 21)是输入/输出点。(b) 停车场内的三个有效移动。每个移动根据所行单元格的数量分配一个权重。}

我们处理的问题类似于Rush Hour难题(图\ref{f2-zh}),这在理论计算机科学领域有相当多的文献。Flake和Baum表明,具有$n \times n$网格的Rush Hour的广义版本是PSPACE-complete。Tromp和Cilibrasi以及Hearn和Demaine表明,仅包含汽车(没有$3 \times 1$的卡车)的Rush Hour也是PSPACE-complete。我们的问题在很多方面都与Rush Hour不同:游戏只允许车辆前后移动,而真实汽车可以转弯;而Rush Hour的目标是移走一个特定的车辆,而真实的停车场必须允许任何请求的车辆到达或离开。在物料处理文献中,也已经讨论了没有定义过道的高密度存储,其中使用可以沿着四个基本方向移动的输送模块存储和检索单元负载或容器。然而,在我们的停车场问题中,汽车独自移动,不需要任何输送模块的帮助,而且汽车可以朝不同的方向行驶。

\section{建模基于谜题的停车场}

我们将停车场视为一个矩形网格。汽车可以水平或垂直停放。我们假设汽车在网格中占据两个单元格,这意味着汽车的大小为$1 \times 2$。停车场的左下角有一个入口/出口点。我们称这个点为输入-输出(I/O)点。在图\ref{f3-zh}a中,两辆车停在一个$4 \times 4$停车场中,一辆停在单元格(43, 44),另一辆停在单元格(24, 34)。单元格(11, 21)表示I/O点。

\fig{f4}{1}{生成$4 \times 4$停车场状态空间图的过程。 (a) 单车图。大红色节点是根节点—这个图中的第一个节点。绿色节点是开放节点。开放节点总是允许新车进入停车场。根节点和一个开放节点都已注释。 (b) $4 \times 4$网格中的2车图。红色节点是初始节点,包括I/O单元格。一个初始节点已注释。 (c) 连接单车图和2车图。单车图中的开放节点连接到2车图中的初始节点,用蓝色突出显示。这些连接器称为桥接边缘。先前注释的节点的位置也显示出来。}

车辆在停车场内可以进行三种类型的移动:直行、直角和平行,这些操作可以前进或后退完成(图\ref{f3-zh}b)。假设车辆花费一个单位的时间在一个单元格上行驶,可以为每个移动分配一个权重以量化行驶时间。例如,车辆进行直行时移动一个单元格,因此将此移动分配一个权重为一。类似地,车辆进行直角和平行移动时移动四个单元格,因此将这些移动分配一个权重为四。我们假设停车场中有一个中央控制器代理,可以与车辆通信以执行所有移动。

读者可能会注意到这里我们并没有指定单元格的尺寸。单元格的大小是根据车辆的尺寸和运动学来决定的。相反,我们希望将单元格的设计留给专业人士。

\subsection{停车场能容纳多少辆车?}

我们使用状态空间图$G = (V, E)$对停车场中的车辆进行建模,其中$V$是表示停车场配置的顶点集合,$E$是表示配置之间可行转换的边缘集合。在本文中,顶点也可以用术语“节点”来表示。配置被定义为停车场的尺寸,停放的车辆数量以及它们的位置。术语“状态”也被用作配置的替代术语。我们想知道:停车场能停放多少辆车?为了开始解决这个问题,我们解决一个小例子,即$4 \times 4$停车场。

在一个$4 \times 4$停车场中,有十六个单元格,每辆车占用两个单元格。因此,我们最多可以放置八辆车。我们首先为$4 \times 4$网格生成状态空间图。在不考虑可操纵性的情况下,一个单独的车辆可以放置在24个不同的位置,每个位置都代表图中的一个潜在节点。然后,我们对所有潜在节点执行两两搜索,确定它们是否可以通过可行的操纵连接。我们将结果称为单车图(图\ref{f4-zh}a)。接下来,我们以每种可能的方式放置两辆车,以创建2车图。然后我们消除了具有重叠车辆的配置。例如,配置[(11, 21), (21, 22)]是不可行的,因为单元格21被两辆车使用,这在物理上是不可能的。我们还消除了占据整行或整列的配置,因为它们是不可行的(除了[(11, 21), (31, 41)])。最后,我们使用它们各自的移动连接节点,构建了完整的2车图(图\ref{f4-zh}c)。同样,我们可以继续构建多达八辆车的图。

我们发现一个有趣的特征来连接这八个图。在一个空的停车场中,当第一辆车到达时,它最初停放在I/O单元格上。这是单车图的起始节点。这辆车可以移动到停车场内的不同位置。一旦车辆离开I/O单元格,停车场就可以接收第二辆车。当第二辆车到达时,它将自己停放在I/O单元格上。这是2车图的起始点。之后,车辆可以重新定位到停车场内的其他位置。类似地,当这两辆车都停在I/O单元格之外的任何位置时,停车场就可以接收第三辆车,我们可以一直进行此过程直到8辆车。因此,我们观察到每个$k$车图($k = 1, 2, \dots, 8$)都从I/O单元格开始。因此,我们将包括占用的I/O单元格的停车场配置定义为初始节点。2车图的初始节点在图\ref{f4-zh}b中以红色突出显示。请注意,每个$k$车图都有多个初始节点,除了1车图。为了将此节点与其他初始节点区分开,我们将其定义为根节点(见图\ref{f4-zh}a中的大红色节点)。我们还观察到为了容纳新车辆进入停车场,我们必须确保没有车辆停在I/O单元格上。这样的配置——I/O单元格未被占用——始终允许新车辆进入停车场。我们将这些配置定义为开放节点(见图\ref{f4-zh}a中的绿色节点)。如果I/O单元格中的任何一个被占用,新车辆将无法进入停车场,这些节点称为非开放节点。在图\ref{f4-zh}a中有五个非开放节点(灰色节点):(11, 21)、(11, 12)、(21, 22)、(21, 31)和(31, 41)。尽管(31, 41)不会立即阻塞停车场,但它只允许第二辆车停在(11, 21)上,之后不能再有车辆进入停车场。

\fig{f5}{0.7}{$4 \times 4$停车场的状态空间图。此图具有5,913个顶点和14,635条边。为了可视化目的,已注释其中一些顶点。}

当第二辆车到达I/O点时,它成为2车图的初始节点。因此,1车图的开放节点和2车图的初始节点通过一种进入车辆的操纵连接在一起。同样,2车图的开放节点和3车图的初始节点也是如此。我们称这些连接器为桥接边缘。1车和2车图的桥接边缘在图\ref{f4-zh}c中用蓝色突出显示。我们一般化,k车图的开放节点和$(k + 1)$-车图的初始节点使用桥接边缘连接。有了桥接边缘,我们就有了一个代表整个状态空间的图(请参见图\ref{f5-zh},显示了$4 \times 4$停车场的状态空间)。

\fig{f6}{1}{基于谜题的检索测试。(a) 一个带有3辆车的$4 \times 4$停车场。(b) 我们在左侧的布局上执行基于谜题的检索测试,并发现通过重新定位停车场内的其他车辆,可以检索出所有三辆车。}

该状态空间图具有5,913个顶点和14,635条边。在此图中,有72个连接组件,即彼此可达的顶点集。来自一个连接组件的顶点与来自其他连接组件的顶点之间不进行通信。

然而,连接组件内的所有顶点都可以相互到达。为了更好地可视化图,我们已注释了一些节点。例如,图\ref{f5-zh}a是一个只有一辆车的停车场,图\ref{f5-zh}d是一个有四辆车的停车场,依此类推。为了确定有多少辆车可以驶入停车场,我们从寻找1车图的顶点开始。1车图始于第一辆车进入停车场的时刻,即根节点。如果我们探索根节点的连接组件,我们可以达到高达7车图的顶点。其余的连接组件不是从根节点开始的。因此,不可能通过驾驶车辆达到这些配置。例如,图\ref{f5-zh}i、图\ref{f5-zh}j、图\ref{f5-zh}k和图\ref{f5-zh}l中的顶点不能从根节点到达。因此,在$4 \times 4$停车场中最多可以停放7辆车。图\ref{f5-zh}g显示了一个有7辆车的停车场。总之,尽管$4 \times 4$停车场有容纳8辆车的能力,但我们只能驾驶7辆车进入。查找我们可以驾驶进入$m \times n$停车场的最大车辆数的步骤如下:

\begin{itemize}
    \item 计算可以放置在停车场中的最大车辆数,$k_{max} = \lfloor\frac{m\times n}{2}\rfloor$。
    \item 为具有$k$辆车的停车场生成状态空间图,其中$k = 1, 2, \dots, k_{max}$。
    \item 对于每个图,识别初始节点和开放节点。
    \item 将$k$车图的开放节点与$(k +1)$车图的初始节点连接起来。这将产生一个具有一组连接组件的图。
    \item 找到带有根节点的连接组件。该组件由1车图到$k_{le}$车图的顶点组成,其中$k_{le}\le k_{max}$。
    \item 返回$k_{le}$的值,即可以驾驶进入停车场的最大车辆数。
\end{itemize}

我们已经了解到,在$4 \times 4$停车场中,我们最多可以停放7辆车。我们只能以后进先出的顺序检索这7辆车。然而,我们并不总是希望为了让另一辆车离开而离开停车场。因此,我们想知道:在停车场中最多可以停放多少辆车,以便其中的任何一辆车在其他车辆保留的情况下离开?为了回答这个问题,我们选择一个有$k$辆车的停车场,其中$k = 1,2,\dots,7$,并测试是否有可能让任何一辆车离开。由于这个测试类似于Rush Hour拼图,我们称之为基于谜题的检索测试。例如,在图\ref{f6-zh}中,我们对一个有3辆车的停车场进行了基于谜题的检索测试。我们观察到在所有三种情况下,目标车辆达到了I/O点,而其余车辆不需要离开停车场,而是重新定位以使目标车辆离开。然后,我们对整个状态空间进行基于谜题的检索测试,以找到可以停放在$4 \times 4$停车场中的车辆数。然而,为了最小化我们的搜索工作量,我们从具有7辆车的停车场配置开始这个测试。我们发现这样停放7辆车是不可能的。然后我们探索具有6辆车的配置,测试再次失败。最后,对于具有5辆车的配置,我们发现所有车辆都可以离开。我们不需要对具有少于5辆车的配置继续进行这个测试,因为肯定会有一些配置通过这个测试。因此,我们发现在$4 \times 4$停车场中,最多可以停放5辆车,以便其中的任何一辆车在其他车辆保留的情况下离开。

在传统设置中,我们希望停放的方式是任何一辆车都可以离开,而不需要重新定位其他车辆。因此,我们想知道:在停车场中最多可以停放多少辆车,以便没有车辆被阻挡?为了回答这个问题,我们选择一个有$k$辆车的停车场,其中$k = 1,2,\dots,5$,并测试是否有可能让任何一辆车离开停车场,而不需要重新定位其他车辆。我们将这个测试称为未阻挡检索测试。然后,通过进行蛮力操作,我们发现在$4 \times 4$停车场中,最多可以停放4辆车,以便没有车辆相互阻挡。

\fig{f7}{1}{三种停车场设计。}
\fig{f8}{1}{$A^*$搜索的单车启发式。为了从左侧停车场中检索深色车辆,右侧停车场被用作启发式,只在与深色车辆相同的位置放置一辆车。(有关本图例中颜色的解释,请参阅本文的网络版本。)}

观察状态空间图,我们发现存在三种停车场设计。第一种类型是停车场,最多停放车辆,车辆完全相互阻挡,车辆只能按照后进先出的顺序离开。我们将这种设计称为有限出口(图\ref{f7-zh}a)。这种停车场通常在不同的活动中发现,如音乐会或球赛。然后是第二种类型,车辆也相互阻挡,但任何车辆都可以通过移动其他车辆离开,就像Rush Hour拼图一样。我们将这种设计称为完全出口(图\ref{f7-zh}b)。最后,第三种是传统停车场(图\ref{f7-zh}c),在这种停车场中,没有车辆相互阻挡。我们发现了这些设计,因为我们在提出研究问题时没有预设任何特定的布局或结构。

\subsection{检索车辆需要多长时间?}

每辆车的检索时间是这些设计的性能指标之一。我们使用相同的状态空间图来计算检索时间。我们在状态空间图上应用$A^*$搜索算法来计算停车场中每辆车的最佳检索。

$A^*$是一种启发式搜索算法,通过扩展根据最小路径成本选择的最有前途的节点来探索图。路径成本评估为$f(s) = g(s) + h(s)$,其中$g(s)$是从初始状态到当前状态s的成本,$h(s)$是从当前状态s到目标状态的最便宜路径的估计成本。Hart等人还表明,如果启发函数$h(s)$是可接受的和一致的,则$A^*$算法保证产生最优结果。一个可接受的启发式$h(s)$是这样构建的,以便它生成到达目标状态的实际成本$h(s^*)$的下界:$h(s)\le h(s^*)$。一致的启发式是这样的,对于每个状态s和s的每个后继状态$s'$,从s到目标的估计成本$h(s)$不大于到$s'$的步骤成本$c(s, s')$加上从$s'$到目标的估计成本$h(s')$,可以表达为$h(s)\le c(s, s') + h(s')$。一致的启发式必然是可接受的,而反之则未必成立。我们已经简要介绍了$A^*$搜索的文献,更多信息请参阅Hart等人,Pearl,Russell和Norvig。

在我们的$A^*$算法实现中,输入包括:起始节点、目标车辆和启发式函数。起始节点是初始状态,由停车场的大小、停放的车辆数量和它们的位置表示。目标车辆是我们要检索的车辆。作为启发式函数,我们使用单车检索时间。例如,在图\ref{f8-zh}中,为了估算从左侧停车场检索深色车辆的成本,我们在右侧停车场的与目标车辆相同的位置放置一辆车,并计算检索时间。深色车辆的检索成本为27(图\ref{f8-zh}a),启发成本为5(图\ref{f8-zh}b)。由于这是我们问题的简化版本,它始终是一致的,因此我们可以得到最优结果。我们将这个启发式函数称为单车启发式。

我们的检索算法通过使用两个列表来探索状态空间:i) frontier(前沿)和ii) explored(已探索)。前沿列表是一个按照$f(s)$值对所有待访问节点进行排名的优先级队列,而已探索列表则存储所有已访问的节点。在开始时,前沿列表只包含初始节点,当选择访问目标节点时搜索停止。目标节点是目标车辆位于I/O点的配置。我们还创建了一个称为父-子的第三个列表,用于跟踪节点之间的关系——每当我们的搜索达到目标节点时,此列表用于构建从初始节点到目标节点的路径。应用$A^*$搜索,我们已经计算出从图\ref{f9-zh}中的有限出口、完全出口和传统停车场检索车辆所需的最佳移动次数。请注意,在有限出口停车场中,车辆的检索遵循后进先出的顺序,我们首先检索第7辆车,然后是第6辆车,依此类推(见图\ref{f7-zh}a),然后计算累积检索时间,如图\ref{f9-zh}a所示。

\fig{f9}{1}{车辆检索时间。}

在这里,我们计算检索时间,假设车辆在一个单元格内行驶需要一个时间单位。然而,读者可能对在真实停车场中的检索时间感兴趣。在这方面,我们在这里展示一个例子。一般而言,一个停车位的尺寸是$9' \times 18'$。由于我们考虑的是一个$1 \times 2$的停车位,我们可以将一个单元格的尺寸视为$9' \times 9'$。根据Schwesinger等人的说法,停车场中自动驾驶车辆的速度限制为$10 km/h (≈ 9ft/s)$。因此,一个单元格的行驶时间为1秒。在完全出口的停车场(图\ref{f9-zh}b)中,为了取回车辆02,所有车辆共行驶27个单元格。因此,车辆02的检索时间为27秒。

\end{document}