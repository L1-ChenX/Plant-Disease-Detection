\renewcommand{\fig}[3]{
	\begin{figure}[!htb]
		\centering
		\includegraphics[width=#2\textwidth]{#1.png}
		\caption{#3}
		\label{#1-zh}
	\end{figure}
}
\renewcommand{\tab}[3]{
	\begin{table}[!htb]
		\centering
		\caption{#3}
		\label{#1-zh}
		\includegraphics[width=#2\textwidth]{#1.png}
	\end{table}
}

\zihao{-4}\songti
\noindent 作者:Tingting Shi, Yongmin Liu, Xinying Zheng, Kui Hu, Hao Huang, Hanlin Liu,Hongxu Huang\\
\noindent 出处:Scientific Reports, 2023, 13(1): 2336.
\vspace{2ex}

\begin{center}
    \zihao{4}\textbf{卷积神经网络在植物病害严重程度评估中的最新进展}
\end{center}


\noindent\textbf{摘要:} 在现代农业生产中,病害严重程度是直接影响植物产量和品质的重要因素。为了对植物的整个生产过程进行有效的监测和控制,不仅要明确病害的种类,还要明确病害的严重程度。近年来,深度学习在植物病害种类识别方面的应用越来越广泛,尤其是卷积神经网络(CNN)在植物病害图像中的应用取得了突破性进展,但对病害严重程度评估的研究相对较少。该研究首先追溯了现有病害研究者的主流观点,为植物病害严重程度的分级提供标准;然后根据网络架构,从经典CNN框架、改进的CNN架构和基于CNN的分割网络三个方面概括了16篇基于CNN的植物病害严重程度评估的研究,并对各自的优缺点进行了详细的比较分析,并研究了数据集获取的常用方法和CNN模型的性能评价指标。最后,本研究讨论了基于CNN的植物病害严重程度评估方法在实际应用中面临的主要挑战,并提出了可行的研究思路和可能的解决方案。

\mySection{引言}
植物病害是造成农业损失的主要原因,这些病害是由各种损害植物生长的生物引起的,例如害虫、细菌或真菌。可靠而准确的病害严重程度评估方法对于有效控制病害和最大限度减少产量损失至关重要。评估植物病害严重程度的方法有很多种。传统的确定病害严重程度的方法是目测评估,但由于病害的相似性和特征的多样性,容易受到外界因素和主观个体差异的影响,这种方法非常不可靠。目测评估通常需要由经验丰富的专家进行,效率不高,而许多农民无法接触到专家,因此很难准确及时地识别病害严重程度。此外,高光谱成像也被用于测量植物病害的严重程度,但这种技术需要传感器等精密设备和一定的专业知识,成本高昂且效率低下。

近年来,随着计算机图像技术的快速发展和相关电子设备硬件性能的不断提升,计算机视觉与人工智能在植物种类分类、叶片病害识别、植物病害严重程度评估等农业诊断领域得到了广泛的应用。深度学习目前在计算机视觉领域取得了重大突破,而CNN在植物病害检测应用中更是表现出了优异的性能。与传统方法相比,CNN能够自动直接从输入图像中提取特征,无需复杂的图像预处理,实现端到端的检测方法。目前,利用CNN进行植物病害种类识别已经取得了令人满意的效果,但在病害严重程度评估方面的研究较少。本研究重点关注CNN在植物病害严重程度评估中的应用,并系统地综述相关研究,为进一步的研究工作提供参考思路。

本综述其余部分安排如下:第二部分概述了与植物病害严重程度视觉评估相关的概念。第三部分回顾了CNN的发展历史。第四部分讨论CNN在植物病害严重程度上的具体应用,说明单任务和多任务系统之间的差异。并从经典CNN框架、改进的CNN架构和基于CNN的语义分割网络三个方面重点介绍基于CNN的植物病害严重程度评估方法的基本工作原理,并分析每种方法的优缺点。第五部分总结了相关的公开数据集并提出了CNN性能评估指标。第六部分讨论了基于CNN的植物病害严重程度评估在实际应用中可能面临的主要挑战,并针对这些挑战提供了可行的研究思路和可能的解决方案。

\mySection{视觉评估}
\mySubsection{植物病害严重程度的定义}
植物病害严重程度定义为可见病害症状的植物单位与植物总单位(例如叶子)的比例,是许多病害的重要定量指标。及时准确地评估病害严重程度对作物生产至关重要,因为病害严重程度直接影响作物产量,并且通常用作预测指标,以极高的准确度估计作物损失。例如,严重程度指标可用作决策阈值或病害预测,以帮助种植者合理化病害控制,例如决定农药的剂量和类型以及喷洒时间。

\mySubsection{植物病害严重程度的视觉评估方法}
准确测量和评估病害严重程度对农业生产至关重要。因为它可以确保正确分析治疗效果、准确了解产量损失与病害严重程度之间的相关性以及合理评估植物生长阶段。不准确或不可靠的病害评估会导致错误的结论,从而导致错误的病害管理措施,从而进一步加剧损失。病害严重程度的评估通常使用各种尺度,包括名义(描述性)尺度、序数评级尺度、区间(类别)尺度和比率尺度。以下概述了这些用于直观评估病害严重程度的尺度,包括定性和定量。

定性尺度。

(1)描述性量表:这是疾病严重程度分级量表中最简单、最主观的标准之一。疾病被分为几类,并用轻度、中度、重度等描述性术语来表示。由于主观性和缺乏量化定义,该量表的价值非常有限,除非在特定情况下进行评级。

(2)定性序数标度:这仍然是描述性病害标度,但病害严重程度的类别比描述性标度更加多样化。例如,徐等人使用0-5的标度来描述西葫芦黄花叶病毒和西瓜花叶病毒的症状严重程度,以表示病害严重程度的增加。该标度被广泛用于某些病害,尤其是用于评估症状不易量化的病毒性疾病。

定量尺度。

(1)定量序数标度:这种标度由已知类别中的数字组成,通常是有症状区域所占的百分比。它可以进一步分为两种类型:等间隔和不等间隔。然而,等间隔评定量度可能会给出更高的平均严重程度,特别是当实际严重程度处于某一类别的低端时,因为间隔太宽,难以显示差异,导致评级不准确。有些病害评定量度的间隔是不等的。Horsfall-Barratt标度(H-B标度)是一种广泛使用的不等间隔标度。它是由Horsfall和Barratt开发的,有效地缓解了等间隔的问题。例如,Bock等人用该标度估计柑橘溃疡病的严重程度。Forbes等人用HB标度估计田间马铃薯晚疫病的严重程度等。

(2)比例量表:该量表广泛用于严重程度的直观评估。评分者测量症状器官的百分比,定义为0\%至100\%,并据此评估严重程度。因此,比例量表对评分者的要求更高,需要更准确地识别和测量实际疾病。
虽然植物病害严重程度可以通过多种不同的方法进行评估,但定性和定量评估方法往往会导致评估结果与现实不符,这是由于评估人员的主观性、在病害严重程度较低时倾向于高估以及评估人员倾向于使用5\%的整数间隔等因素造成的。为了提高评估人员估计的准确性,标准面积图(SAD)长期以来一直被用作帮助评估植物病害严重程度的工具。对评估人员进行专业培训也可以有效提高评估的准确性。


\mySection{CNN的发展历史}
深度学习始于1943年阈值逻辑的提出,本质上是建立与人类神经网络极为相似的计算机模型的过程。CNN是深度学习的一个子集,出现于20世纪80年代,最初发展了感受野的概念,随后被引入到CNN研究中,后来随着BP算法的引入和多层感知器的训练,研究者们尝试自动提取特征,而不是手工设计特征。LeCun et al.提出了一种使用BP网络的CNN架构“LeNet-5”,在当时标准的手写数字识别任务上,其表现优于所有其他技术。由于传统BP神经网络随着网络层数的增加,出现了局部最优、过拟合、梯度消失等问题,以及当时一些浅层机器模型的提出,深度神经网络模型的研究一度被搁置。直到2006年左右,Hinton等人发现具有多个隐藏层的人工神经网络具有出色的特征学习能力,Glorot等人通过一种归一化方法缓解了训练过程中梯度消失的问题。人们的注意力又转移到深度学习上。2012年,AlexNet赢得了ImageNet大规模视觉识别挑战赛(ILSVRC),自此DL受到了越来越多研究者的关注,AlexNet被认为是深度学习领域的重大突破。接下来,CNN架构不断发展,涌现出许多性能优异的算法。主要的经典CNN网络有LeNet、AlexNet、VGG、GoogLeNet、Resnet、DenseNet等,从LeNet到DenseNet的演进序列顺序如图\ref{f2-1-zh}所示。

\fig{f2-1}{0.5}{CNN从LeNet到DenseNet的演进时间线}

\mySection{基于CNN的植物病害严重程度评估方法}
CNN在评估植物病害严重程度方面已取得巨大成功。基于CNN的植物病害严重程度自动评估首次由Wang等人于2017年提出。他们使用不同的CNN模型对四个严重程度等级的苹果黑腐病图像进行分类,在测试集上取得了90.4\%的总体准确率,这表明CNN是一种很有前途的全自动植物病害严重程度分类新技术。Liang等人提出了PD2SE-Net,实现了一个用于病害严重程度评估、植物物种识别和植物病害分类的多任务系统,总体准确率分别为0.91\%、0.99\%和0.98\%。Su等人结合ResNet-101网络和语义分割,快速预测小麦赤霉病(FHB)的严重程度,预测准确率为77.19%。

\mySubsection{单任务与多任务系统}
深度学习倾向于针对特定指标进行优化。换句话说,一个模型或一组模型通常被训练来执行单个目标任务,这样的系统被称为单任务系统。另一方面,还有多任务学习(MTL)的概念,如果将多个任务链接在一起,则可以同时学习。实验研究表明,在预测性能方面,同时从多个相关任务中学习特征比单独学习它们更有益。MTL可以通过并行学习任务并使用来自不同任务的更多特征来降低每个任务中过度拟合的风险,从而实现更好的模型泛化。

利用CNN进行植物病害检测的研究包括单任务系统,可单独识别植物病害种类或估计病害严重程度。例如,Prabhakar等人使用ResNet101评估番茄叶枯病的严重程度。Zeng等人训练了六种不同的CNN模型来对柑橘黄芽病的严重程度进行分类。也有多任务系统同时执行这两项任务。例如,José GM Esgario等人使用CNN实现了咖啡叶病种类的分类和严重程度分级。Fenu等人考虑了五种预训练的CNN架构作为特征提取器,对三种疾病和六种严重程度进行分类,其实验结果表明,训练后的模型在使用多任务学习模型自动提取病害叶片识别特征方面具有很强的健壮性。


\mySubsection{CNN在植物病害严重程度评估中的应用}
为了明确CNN用于植物病害严重程度评估的具体实现过程,本研究选取了16篇符合研究主题的高质量文章。首先,在世界最大、最全面的科学信息资源之一Web of Science平台上进行搜索。收集研究集的过程需要定义搜索词,因此在Web of Science中输入关键词“卷积神经网络”(主题)和“植物病害严重程度”(主题),截至2022年,共检索到57篇文章,出版年份如图\ref{f2-2-zh}所示。在57篇论文中,根据研究对象(植物病害)和研究方法(CNN)选取16篇论文进行具体分析。在此基础上,对2022年的最新研究进行单独分析。根据这16篇文章采用的CNN网络架构的不同,进一步分为三类:经典CNN框架、改进的CNN架构和基于CNN的分割网络。基于CNN的植物病害严重程度评估方法流程图如图\ref{f2-3-zh}所示。

\fig{f2-2}{0.7}{以“卷积神经网络”和“植物病害严重程度”为关键词的57篇文章发表年份分布图}

\fig{f2-3}{0.7}{基于CNN的植物病害严重程度评估方法流程图}

\mySubsection{经典CNN框架}
16篇文章中有10篇是基于经典CNN框架实现病害严重程度分级的,这10篇研究在具体的CNN框架和研究主题上有所不同,但都是基于CNN实现植物病害严重程度评估,因此在具体实现过程上有相似之处。基于经典CNN框架实现植物病害严重程度评估的过程可以分为以下三个主要步骤。

第一步是收集和处理数据集,从四个方面进行描述。

(1)数据集特点。10篇研究中,6篇使用自制数据集,4篇使用PlantVillage图片。自制数据集又可分为两类,一类是在受控条件下拍摄的图像,照片是在部分受控条件下从叶片背面拍摄的,背景为白色;另一类是在自然条件下拍摄的,背景复杂,而PlantVillage中的图像背景均匀、均质。自制数据集耗时长、成本高,但更符合真实环境。大量研究表明,用受控图像训练的模型预测真实环境图像时,准确率会显著降低。如果公开的数据集不能满足某项研究的需求,就必须自制数据集。

(2)数据集标注。评估严重程度的必要条件之一是记录被标注不同的严重程度等级。10篇文章中,3篇按描述性尺度标注,1篇按定性序数尺度标注,4篇按定量序数尺度标注,1篇文章未在文章中注明标注方法。例如,在46篇文章中,采用了定量序数尺度。根据患病叶片的比例将严重程度分为五个等级:健康(<0.1\%)、非常低(0.1–5\%)、低(5.1–10\%)、高(10.1\%–15\%)和非常高(>15\%)。

(3)数据集划分。数据集通常分为三部分:训练数据集、验证数据集和测试数据集。训练集用于训练模型,验证集用于调整超参数,测试集用于评估模型性能。这10项研究基本上都使用了70\%到85\%的数据集进行训练。Mohanty等人尝试了五种不同的分离比例来划分数据集,实验结果表明,对他们的数据来说,使用80\%的数据集进行训练,20\%的数据集进行验证是理想的。

(4)数据预处理。通常,在将图像输入CNN之前会进行两项预处理操作。一是调整图像大小以匹配输入层要求。例如,PlantVillage中的图像大小为256×256,而AlexNet输入层要求大小为227×227,则需要调整原始照片的大小。此处理在所有10项研究中均有体现。其次,对图像进行归一化,以帮助模型更快地收敛,显著提高端到端训练的效率。

第二步是模型选择和训练阶段,下面从两个方面进行描述。

(1)CNN框架选择。10项研究中使用的CNN框架包括AlexNet、VGG、GoogLeNet、ResNet、DenseNet、MobileNet、Inception、Faster R-CNN、YOLO、EfficientNet、SqueezeNet、Xception等。这些研究中绝大多数都使用了多个CNN框架进行对比实验,以确定在相同训练条件下哪种模型更能检测出某种植物疾病的严重程度。

(2)训练方法。训练CNN有两种方法,一种是从头开始,另一种是迁移学习。迁移学习是指将一个在大量图像集(如ImageNet(1000个类别,120万张图像))上训练好的网络适应不同的任务,这是通过底层CNN学习非特定特征来实现的。迁移学习有两种方法:特征提取和微调。特征提取是保持预训练模型的权重不变,然后使用它们在目标数据集上训练新的分类器的过程。微调涉及使用预训练模型中的权重初始化模型,然后在目标数据集上训练部分或全部权重。Brahimi等人使用了特征提取、微调和从头训练三种方法来训练六个CNN模型。结果表明,微调模型的准确率最高,特征提取模型的训练时间最短。10篇研究中有8篇采用了迁移学习,只有一篇是从头开始训练的。在44篇论文中,对两种训练模型的方法进行了比较,结果表明迁移学习缓解了训练数据不足的问题。


\mySubsection{最后一步是评估CNN模型的性能}
CNN模型的性能是通过对训练好的模型使用测试集来获得的。测试集必须独立于训练集和验证集,否则评估结果可能会有很大偏差。Mohanty等人训练了一个模型来识别14种作物和26种疾病,在测试集上的总体准确率为99.35\%,其中验证集和测试集之间没有明显的区分。当他们在一组与训练图像不同条件下拍摄的图像上测试该模型时,模型的准确率急剧下降到31\%。值得注意的是,10项研究中只有4项明确区分了三类数据集。Sibiya等人明确区分了训练集、验证集和测试集。实验结果表明,所提出的模型既没有过度拟合也没有欠拟合,因为该模型在验证集上的准确率为95.63\%,在测试集上的准确率高达89\%。

模型性能的定量评估是通过评估指标实现的。评估指标通常包括准确率、精确率、召回率、平均精确率(mAP)以及基于精确率和召回率的F1分数。随着深度学习的发展,CNN模型在不同数据集上的性能得到了提高,各种评估指标也得到了增加。很难对来自不同研究的CNN进行一致的性能比较,因为大多数基于CNN的植物病害严重程度评估研究都适用于特定数据集,其中许多数据集尚未公开,并且没有提供重现实验所需的所有参数。

\mySubsection{改进的CNN架构}
16篇文章中有2篇是基于改进的CNN架构进行严重程度评估的。将经典的CNN框架与改进的CNN架构进行比较,相同之处在于实现过程基本相同,不同之处在于后者采用基于经典CNN的改进网络,目的是设计出性能更高、更实用的植物病害诊断系统。

提出了一个网络PD2SE-Net,以设计一个更优秀、更实用的植物病害诊断系统。PD2SE-Net引入了ResNet50网络作为基础模型,并集成了ShuffleNet-V230的构建块。PD2SE-Net架构如图\ref{f2-4-zh}所示。PD2SE-Net有两个关键组件使其如此有效。一是引入残差结构来构建参数共享层,这使得模型每批可以更新更多信息。受ShaResNet66的启发,ResNet50被用来构建基本框架,并与参数共享相结合,以减少网络中的冗余信息。另一个是引入了shuffle单元。ShuffleNet-V2单元用于以较低的计算复杂度提取不同植物种类和疾病的特征图。最后,PD2SE-Net实现了植物种类识别、疾病分类和严重程度估计,总体准确率分别为0.99、0.98和0.91。

\fig{f2-4}{0.7}{PD2SE-Net的架构。分为五部分:(a)参数共享层;(b)第三层是第四层与混洗块2之间的参数共享层,第四层是用于严重程度估计的高维特征提取器;(c)用于植物物种识别的特征提取器;(d)用于植物疾病诊断的特征提取器;(e)全连接层3}

Xiang等提出了一种基于残差网络、通道重排操作和多尺寸模块的轻量级网络L-CSMS,用于植物病害严重程度评估。多尺度卷积模块使用多个不同大小的卷积核来提取不同的感受野,从而从特征图中获得鲁棒的特征和空间关系。引入通道重排操作以实现不同通道组之间的信息通信并提高精度。通道重排操作和多尺寸卷积模块以堆叠拓扑的形式集成到构建块中,如图\ref{f2-5-zh}所示。L-CSMS使用ResNet的残差学习方法,通过堆叠相同拓扑的模块来构建深度网络。为了验证L-CSMS模型的性能,Xiang等将L-CSMS模型与ResNet、DenseNet、Inception-V4、PD2SE-Net、ShuffleNet和MobileNet进行了对比实验。结果表明,L-CSMS以更少的参数、FLOP和较好的精度取得了竞争优势。

\fig{f2-5}{0.7}{具有通道混洗操作和多尺寸卷积模块的构建块}

\mySubsection{基于CNN的语义分割网络}
图像语义分割越来越受到计算机视觉和深度学习研究人员的关注,使用深度学习技术进行语义分割的研究工作也在不断发展。特别是,CNN在准确性和效率方面远远超过了其他方法。基于CNN的分割不仅提供类别信息,还提供关于这些类别的空间位置的附加信息。语义分割的任务是将每个像素标记为一种封闭对象或一类区域。基于CNN的分割理论已经应用于植物病害严重程度估计和农业中的其他相关研究。语义分割应用于植物病害严重程度估计的主要目标是给每个像素分配合适的标签,以获得病害严重程度估计所需的患病区域百分比。

通常,语义分割的架构分为两部分:编码器网络和解码器网络。编码器通常基于CNN网络,生成低分辨率图像表示或特征图,映射到像素级图像,然后进行预测和分割。不同的语义分割模型之间的差异通常体现在解码器网络上。深度学习在语义分割中的第一个成功应用是由Long等人构建的全卷积网络(FCN)。此后,出现了许多语义分割的变体,例如U-Net、SegNet、DeepLab等。

16篇文章中有4篇使用基于CNN的语义分割网络进行植物病害严重程度评估。陈等提出了一种用于评估水稻细菌性叶纹病(BLS)严重程度的BLSNet。BLSNet在U-Net的基础上加入了注意力机制和多尺度提取以提高病斑分割的准确性。与DeepLabv3+和U-Net相比,实验结果表明BLSNet更能适应图像的尺度变化,BLSNet的预测时间略长于U-Net,但短于DeepLabV3+。高等提出了一种基于SegNet的网络来分割马铃薯晚疫病(PLB)病斑,以量化PLB的严重程度。Goncalves等人对六种语义分割网络(U-net、SegNet、PSPNet、FPN和DeepLabv3+的2个变体)应用于三类植物病害严重程度估计(咖啡潜叶虫病、大豆锈病和小麦褐斑病)进行了比较实验。

虽然CNN在植物病害严重程度评估方面已经取得了不错的效果,但是基于CNN的语义分割网络也有其优势。CNN模型在植物病害严重程度评估方面的成就在于直接建立了严重程度与样本之间的关系,这种关系对某些植物病害适用,但对另一些病害可能并不适用,对于其他病害,模型需要重新训练。基于CNN的语义分割网络通过像素级分割得到病害叶面积百分比来反映严重程度,很好地解决了这一问题。前期研究已经证明了基于CNN的语义分割网络在植物病害严重程度评估方面的可行性。

\mySubsection{2022年的研究}
在2022年的研究中,利用CNN评估植物病害严重程度的方法大致可以分为两类,一类是基于分割的,另一类是基于改进CNN,具体来说是增加了Attention机制。在分割评估方法中,常用的分割网络有DeepLabV3+、U-Net、PSPNet和Mask R-CNN等。例如,张等采用三阶段的方法对黄冠梨进行分级,第一阶段利用Mask R-CNN从复杂背景中分割出黄冠梨,第二阶段利用DeepLabV3+、U-Net和PSPNet对黄冠梨斑点进行分割,并计算斑点面积与黄冠梨像素面积的比值,分为三个等级;第三阶段采用ResNet-50、VGG-16和MobileNetV3得到黄冠梨的等级。刘等也借鉴了阶段分割的思想,首先利用深度学习算法将苹果叶片从复杂背景中分割出来,然后对分割后的叶片进行病害面积识别,计算病害面积与叶片面积的比值来评估病害严重程度。实例分割可以有效地将目标从复杂背景中分离出来,有利于处理真实环境。在另一种方法中,注意力机制(Attention Mechanism)引起了人们的关注,尹等在加入多尺度和注意力机制的基础上对DCNN进行了改进,实现了玉米小叶斑病的分类。刘等在SqueezeNext78中引入了多尺度卷积核和坐标注意力机制来估计病害严重程度,比原SqueezeNext模型提高了3.02%。

\mySection{数据集和评估指标}
\mySubsection{植物病害严重程度数据集}
正确构建和合理使用植物病害严重程度数据集是开展严重程度评估工作的前提和基础。与计算机视觉领域的ImageNet、PlantVillage、COCO等不同,植物病害严重程度数据集尚无统一的大型数据集。植物病害严重程度数据集可以通过自己拍照并标注图片,也可以利用公开的数据集然后标注图片并引用他人标注的图片来收集。随着电子设备的发展和普及,图像采集一般通过摄像头和智能手机进行。PlantVillage是植物病害严重程度领域常用的公开数据集,常用图像标注软件有LabelMe、LabelImg等。本节提供16篇研究中使用的数据集及标注软件链接,如表\ref{t2-1-zh}所示。

\tab{t2-1}{0.7}{在这16项研究中使用的数据集和标注软件:A代表经典CNN框架,B代表改进型CNN架构,C代表基于CNN的语义分割网络,D代表标注软件}

\mySubsection{评估指标}
前面模型性能评估中提到的常见评估指标有准确率、精确率、召回率、平均精确率(mAP),以及基于精确率和召回率的F1分数,下面分别介绍它们的具体定义。

准确率、精确率和召回率由以下公式表示:

\begin{align*} Accuracy = \frac{TP + TN}{{TP + TN + FP + FN}} \cdot 100\% \tag{1}\\ Precision = \frac{TP}{{TP + FP}} \cdot 100\% \tag{2}\\ Recall = \frac{TP}{{TP + FN}} \cdot 100\% \tag{3}\end{align*}

在公式(1)和公式(2)中,真实阳性(TP)表示正确识别的病变数量,预测值为1,实际值为1。假阳性(FP)表示错误识别的病变数量,预测值为1,实际值为0。假阴性(FN)表示未识别的病变数量,预测值为0,实际值为1。真实阴性(TN)表示正确识别的非病变数量,预测值为0,实际值为0。

首先需要计算数据集中每个类别的平均精度,得到mAP。

\begin{equation*} P_{average} = \sum\nolimits_{j = 1}^{{N\left( {class} \right)}} {Precision\left( j \right) \cdot Recall\left( j \right) \cdot 100\% }\tag{4}\end{equation*}

在上式中,N是所有类别的数量,j是数据集中的具体类别。

每个类别的平均精度定义如下:

\begin{equation*} mAP = \frac{{P_{average} }}{{N\left( {class} \right)}}\tag{5}\end{equation*}

F1分数同时考虑了模型的准确率和召回率,其公式为:

\begin{equation*} F1 = \frac{2Precision \cdot Recall}{{Precision + Recall}} \cdot 100\%\tag{6}\end{equation*}

\mySection{挑战与未来展望}
传统机器学习(ML)方法中最重要的和最耗时的部分之一是手动特征提取,而CNN可以自动学习特征。Hedjazi MA等人通过预训练CNN模型解决了图像中叶子的视觉识别任务。实验结果表明,预训练的CNN模型优于使用局部二值模式(LBP)的传统机器学习方法。Bhujel A等人设计并测试了基于深度学习的语义分割模型,以检测和测量草莓植株上的灰霉病。结果表明,Unet模型在检测和量化灰霉病方面优于传统的XGBoost、K-means和图像处理技术。与传统图像处理方法和机器学习相比,无论是通过经典的CNN框架、改进的CNN架构还是基于CNN的语义分割网络,植物病害严重程度评估都具有广阔的应用前景和巨大的发展潜力。植物病害严重程度评估技术虽然发展迅速,并逐渐从学术研究走向农业应用,但距离现实自然环境中的成熟应用还有一定差距,还有很多问题需要解决。

\mySubsection{数据集问题}
数据集问题主要可以分为两个方面:数据集不足和数据集不平衡。

(1)数据集不足。充足的数据集是训练网络的必要和基础。然而,收集和构建数据集是一个极其耗时、费力且成本高昂的过程。虽然有许多关于植物病害的公开数据集,如PlantVillage、ImageNet,以及一些公开的自制数据集。然而,植物病害严重程度研究确实需要严重程度注释的数据集。图像的严重程度注释是一个更繁琐的过程。在注释过程中要面对两个问题,一个是效率问题,另一个是准确性问题。为了解决手工注释中出现的耗时和复杂的注释过程,一个可能的解决方案是使用先进的软件实现注释的自动化,这种自动注释算法是迫切需要的。此外,半监督训练和辅助标记方法可用于提高农业样品处理的速度,并有助于减轻手工语义标记的工作量问题。对于准确性问题,无论是通过人工视觉评估还是通过软件进行注释,错误都是不可避免的,这是未来研究的挑战。

(2)数据集不平衡。不平衡问题会对模型的性能产生严重影响,例如,误分类率会变得更高,这在许多研究的实验中已经得到证实。数据增强和加权损失函数81可以很好地缓解这一问题。

\mySubsection{复杂的背景问题}
根据图像背景,数据集可分为两类:在受控条件下拍摄的具有均匀背景的图像和在自然环境中拍摄的具有复杂背景的图像。与均匀背景相比,使用在自然环境中拍摄的图像进行训练可以使CNN模型更具泛化能力。同时,图像中的复杂背景还会带来其他负面问题。例如,在现实环境中,地面污渍与疾病症状相似,从而导致模型出现分类错误。自然光的反射会导致阴影健康区域被错误分类或无法检测到疾病区域。此外,在现实环境中,多种疾病同时发生的情况更为常见。研究人员提到,一片叶子上同时存在多种疾病会显著改变症状的特征,尤其是当症状重叠时,使系统更容易出现错误分类。众多研究表明,当病害症状相似时,它们的错误分离率会显著增加。由于复杂背景带来的问题,基于CNN的植物病害严重程度评估的理论结果应用到实际农业生产过程中面临严重的障碍。为了解决一些复杂环境带来的问题,可以对图像进行预处理,但这增加了整个检测过程的复杂性。针对多种病害同时识别评估的问题,研究人员提出通过训练一种基于相似性的架构来缓解这一问题,该架构将数据集中与任何病害都不相似的症状归为新的类,例如其他类。这个想法还没有实现,需要集思广益寻找进一步的解决方案。

\mySubsection{实用性问题}
为了将理论研究应用到实际情况中,人们提出了各种各样的解决方案。众所周知,DCNN是一种有效的自主特征提取模型。一些研究将深度学习和机器学习方法结合起来建立混合模型。通常,CNN用作特征提取部分,机器学习方法用作分类器。Saberi Anari等人使用改进的CNN进行特征提取。并使用多个支持向量机(SVM)模型来提高特征识别和处理的速度。Kaur等人使用EfficientNet-B7进行特征提取。经过迁移学习后,他们使用逻辑回归技术对收集到的特征进行采样。最后,使用提出的方差技术从特征提取向量中去除不相关的特征。并使用分类算法对得到的特征进行分类,以98.7\%的最高恒定准确率识别出最具判别力的特征。通过消除不相关的特征,模型的参数大大减少。Vasanthan等采用AlexNet和VGG-19进行特征提取,通过相关系数选取最佳特征子集,并输入到K近邻、SVM、脉冲中子网(PNN)、模糊逻辑、人工神经网络(ANN)等分类器中,实验结果表明该方法的平均准确率在96\%以上。

需要一台具有超强计算能力的服务器来确保实验室中CNN构建的植物病害严重程度模型得到广泛应用。云计算本质上是一个共享的计算资源池。云计算将众多计算资源聚集起来,通过软件实现自动化管理。不受时间和空间的限制,任何使用互联网的人都可以使用网络上庞大的计算资源和数据中心。PaaS云是云计算的具体实现。PaaS提供商提供许多基础设施和其他IT服务,用户可以通过Web浏览器在任何地方访问它们。按使用付费的能力使组织可以消除传统上用于本地硬件和软件的资本支出。Lanjewar等在PaaS云中部署了用于评估茶叶病害的CNN模型,智能手机可以访问已部署模型的超链接。茶叶的图像可以通过智能手机摄像头拍摄并上传到云端。云系统自动预测疾病并将其显示在手机显示屏上。Lanjewar MG等使用PaaS云平台部署了用于姜黄检测的CNN模型。云计算在给我们带来便利的同时,也继承了计算机和互联网所共有的安全问题,特别是隐私问题、资源窃取、攻击、计算机病毒等,这些潜在的安全问题十分严重,值得我们重视。

在云计算平台上部署CNN模型,模型尺寸越小越好,但模型越小是否能达到同样的评测效果是一个值得讨论的问题。增加模型尺寸到一定程度,特征提取效果会更好,比如AlexNet与VGG、GoogLeNet等DCNN模型的对比,但随着模型越来越深越来越大,会出现退化问题,ResNet的残差结构可以有效缓解这一问题。越来越多的轻量级网络被提出,它们的效率可能不是最好的,但值得用少量的有效性换取大量的效率。刘等对SquezeNext进行了改进,并与ReseNet-50、Xception、MobileNet-V2进行了对比实验,实验结果表明,所提方法的准确率略优于Xconcept,而模型大小只有2.83MB,仅为Xconcept的3.45\%。模型结构是平衡模型大小和性能的关键因素。

尽管CNN在评估植物疾病严重程度方面表现出色且潜力巨大,但CNN也有其自身的局限性,例如平移不变性、池化层导致信息丢失以及无法很好地获取全局特征。作为可能的贡献和未来的工作,最近变得相当流行的新技术,例如视觉转换器。视觉转换器的主要特点是自注意力机制,可以很好地捕捉全局信息。据我所知,还没有研究将其应用于严重程度估计。

目前,有些问题仍然没有找到合适的解决方案,这表明目前植物病害严重程度自动评估的研究还远远没有达到成熟和完善的实际应用,需要更多的学者继续努力研究尚未解决的问题。本课题组关于“利用卷积神经网络进行植物病害严重程度评估的最新进展”的综述文章为相关类型的研究工作提供了一些参考,更重要的是可以为后续的研究工作提供新的思路。
